\section{Der elektrische Strom}
In der Elektrostatik beschäftigten wir uns zunächst mit statisch ruhenden Teilchen. Die besondere Eigenschaft war hier gerade, dass längs eines Leiters keine Potentialdifferenz auftrat.
\par Jede Bewegung von elektrischen Ladungen bezeichnen wir als einen \textbf{elektrischen Strom}, dieser ist definiert durch die Bewegung positiver Ladungsträger.
Wie kommt es jedoch zu diesem \textbf{Teilchentransport}, der die Quelle jeder Teilchen sein sollte? In den meisten Fällen handelt es sich hier um externe Spannungsquellen, welche je nach Polung einen Elektronenmangel bzw. Elektronenüberschuss erzeugen können.
Diese Elektromotorische Kraft (EMK) hält die Bewegung der Teilchen aufrecht. Ohne diese externe \textbf{Potentialdifferenz }fließen die Elektronen nach einiger Zeit nicht mehr und ruhen. Genau dann befinden wir uns im elektrostatischen Zustand.

\subsection{Stromstärke und Stromtransport}

\paragraph{Elektrischer Strom}
Ein elektrischer Strom beschreibt die Anzahl an Ladungen $dQ$ , die pro Zeitintervall $dt$ einen Leiterquerschnitt durchfließen. Allgemein ergibt sich
\begin{align} \label{def:Strom}
\boxed{I := \frac{dQ}{dt}}
\end{align}
Die Einheit der Stromstärke ist festgelegt durch $1\mathrm{A(mpere)}$. Ampere ist eine Basisgröße im SI-System lässt sich folglich nicht durch andere Größen darstellen. 

Für Gleichströme können wir uns überlegen, dass die Ladungsmenge endlich gegeben ist
 mit $Q = n\cdot q$t (Ladung kann schließlich nur in Quanten auftreten). Es folgte für das Zeitintervall $T$ 
 \begin{align} \label{eqn:elektrischer Strom Gleichspannung}
I_{gl} = \frac{n\cdot q}{T}
\end{align}

\subsection{Kirchhoffsche Regeln}
Netzwerke von Schaltungen können unterschiedlich aufgebaut sein. Man unterscheidet zwischen so genannten Knoten- und Maschentopologien. 
\subsubsection{Knotenregeln}
In Knoten kann keine Ladung gespeichert werden, so sollte die \textbf{Summe der zufließenden Ströme den abfließenden }entsprechen. 
\begin{align} \label{eqn:Knotenregel}
\boxed{\sum \limits_{k=1}^n I_k = 0}
\end{align}

\subsubsection{Maschenregel}
Ein zweiter möglicher Aufbau ist eine Masche. Umläuft man einmal eine gesamte Masche, so muss die \textbf{Summe aller Potentialdifferenzen null} ergeben. Hierbei ist es unabhängig, wie der Umlaufsinn festgelegt wurde.
Die Wichtung von Klemmspannung und den Spannungsabfällen an Widerständen (etc.) erfolgt anti-parallel, also mit unterschiedlichem Vorzeichen. Die Kirchhoffschen Regeln kann man sich bildlich als Vektoraddition hin zum Nullvektor vorstellen.

\begin{align} \label{eqn:Maschenregel}
\boxed{\sum \limits_{k=1}^n U_k = 0}
\end{align}

\subsection{Stromdichte}
Stromdichte $\vec{j}$ ist eine vektorielle Größe und beschreibt den Ladungstransport durch einen Leiter mit einem bestimmten Querschnitt A. Für gutartige Leiter gilt demnach
 \begin{align*}
j = \frac{I}{A}
\end{align*}
Besser ist jedoch die vektorielle Aufsummierung über die Flächennormalen von Querschnittsflächen. 
\begin{align} 
\boxed{I = \int \limits_A \vec{j} \cdot d\vec{A}}
\end{align}
Hauptträger der elektrischen Ladung in el. Leitern sind vorwiegend Elektronen, weitere Vertreter sind Ionen die z.B: Ladungsaustausch in Ionengittern (Elektrolyten etc.) ermöglichen. In Plasmen treten nicht nur spannende Effekte wie z.B. die Polarlichter auf, hier werden zugleich Ionen und Elektronen für den Austausch von Ladungen benötigt.

% Diese Def. ist nicht mehr zeitgemäß!
\begin{align} \label{eqn:Stromdichte}\begin{split}
\vec{j} &= nq\vec{v} \\
\vec{j}	&= \varrho_{el}\vec{v}
\end{split}
\end{align}
Mit der Ladungsdichte geschrieben als $\varrho_{el} = nq$ (bemerke: Die Ladungsdichte nicht mit Dichten von Massenverteilungen bzw. dem elektrischen Widerstand verwechseln!)



\end{align*}

\begin{align} \label{eqn:}
•
\end{align}
