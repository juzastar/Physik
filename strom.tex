\section{Der elektrische Strom}
In der Elektrostatik beschäftigten wir uns zunächst mit statisch ruhenden Teilchen. Die besondere Eigenschaft war hier gerade, dass längs eines Leiters keine Potentialdifferenz auftrat.
\par Jede Bewegung von elektrischen Ladungen bezeichnen wir als einen \textbf{elektrischen Strom}, dieser ist definiert durch die Bewegung positiver Ladungsträger.
Wie kommt es jedoch zu diesem \textbf{Teilchentransport}, der die Quelle jeder Teilchen sein sollte? In den meisten Fällen handelt es sich hier um externe Spannungsquellen, welche je nach Polung einen Elektronenmangel bzw. Elektronenüberschuss erzeugen können.
Diese Elektromotorische Kraft (EMK) hält die Bewegung der Teilchen aufrecht. Ohne diese externe \textbf{Potentialdifferenz }fließen die Elektronen nach einiger Zeit nicht mehr und ruhen. Genau dann befinden wir uns im elektrostatischen Zustand.

\subsection{Stromstärke und Stromtransport}

\paragraph{Elektrischer Strom}
Ein elektrischer Strom beschreibt die Anzahl an Ladungen $dQ$ , die pro Zeitintervall $dt$ einen Leiterquerschnitt durchfließen. Allgemein ergibt sich
\begin{align} \label{def:Strom}
\boxed{I := \frac{dQ}{dt}}
\end{align}
Die Einheit der Stromstärke ist festgelegt durch $1\mathrm{A(mpere)}$. Ampere ist eine Basisgröße im SI-System lässt sich folglich nicht durch andere Größen darstellen. 

Für Gleichströme können wir uns überlegen, dass die Ladungsmenge endlich gegeben ist
 mit $Q = n\cdot q$t (Ladung kann schließlich nur in Quanten auftreten). Es folgte für das Zeitintervall $T$ 
 \begin{align} \label{eqn:elektrischer Strom Gleichspannung}
I_{gl} = \frac{n\cdot q}{T}
\end{align}

\subsection{Kirchhoffsche Regeln}
Netzwerke von Schaltungen können unterschiedlich aufgebaut sein. Man unterscheidet zwischen so genannten Knoten- und Maschentopologien. 
\subsubsection{Knotenregeln}
In Knoten kann keine Ladung gespeichert werden, so sollte die \textbf{Summe der zufließenden Ströme den abfließenden }entsprechen. 
\begin{align} \label{eqn:Knotenregel}
\boxed{\sum \limits_{k=1}^n I_k = 0}
\end{align}

\subsubsection{Maschenregel}
Ein zweiter möglicher Aufbau ist eine Masche. Umläuft man einmal eine gesamte Masche, so muss die \textbf{Summe aller Potentialdifferenzen null} ergeben. Hierbei ist es unabhängig, wie der Umlaufsinn festgelegt wurde.
Die Wichtung von Klemmspannung und den Spannungsabfällen an Widerständen (etc.) erfolgt anti-parallel, also mit unterschiedlichem Vorzeichen. Die Kirchhoffschen Regeln kann man sich bildlich als Vektoraddition hin zum Nullvektor vorstellen.

\begin{align} \label{eqn:Maschenregel}
\boxed{\sum \limits_{k=1}^n U_k = 0}
\end{align}

\subsection{Stromdichte}
Stromdichte $\vec{j}$ ist eine vektorielle Größe und beschreibt den Ladungstransport durch einen Leiter mit einem bestimmten Querschnitt A. Für gutartige Leiter gilt demnach
 \begin{align*}
j = \frac{I}{A}
\end{align*}
Besser ist jedoch die vektorielle Aufsummierung über die Flächennormalen von Querschnittsflächen. 
\begin{align} 
\boxed{I = \int \limits_A \vec{j} \cdot d\vec{A}}
\end{align}
Hauptträger der elektrischen Ladung in el. Leitern sind vorwiegend Elektronen, weitere Vertreter sind Ionen die z.B: Ladungsaustausch in Ionengittern (Elektrolyten etc.) ermöglichen. In Plasmen treten nicht nur spannende Effekte wie z.B. die Polarlichter auf, hier werden zugleich Ionen und Elektronen für den Austausch von Ladungen benötigt.

% Diese Def. ist nicht mehr zeitgemäß!
\begin{align} \label{eqn:Stromdichte}\begin{split}
\vec{j} &= nq\vec{v} \\
\vec{j}	&= \varrho_{el}\vec{v}
\end{split}
\end{align}
Mit der Ladungsdichte geschrieben als $\varrho_{el} = nq$ (bemerke: Die Ladungsdichte nicht mit Dichten von Massenverteilungen bzw. dem elektrischen Widerstand verwechseln!)

\subsection{Ohmsche Widerstände}
Durchlaufen Elektronen einen ohmschen Leiter, so finden fortlaufend Stoßprozesse statt, die zur Verlangsamung selbiger führen. Gitterschwingungen im Leiter selber sind die überwiegenden Effekte, die zu Reibungseffekten führen können.

\subsubsection{Ohmsches Gesetz}
Viele dürften mit dem Begriff des Ohmschen Gesetzes die Gleichung $U = R\cdot I$ verbinden. Leider ist dies ein großer Irrtum, dass Ohmsche Gesetz beschreibt die Proportionalität von homogenen ohmschen Widerständen zur Leiterlänge und die Proportionalität zum Kehrwert des Leiterquerschnitts.
\begin{align} \label{eqn:Ohmsches Gesetz}
\boxed{R = \varrho_{el} \frac{l}{A}}
\end{align}Wobei $\varrho_{el}$ den spezifischen Widerstand beschreibt ($\frac{1}{\varrho_{el}} =  \sigma$) Die Einheit von ohmschen Widerständen wir angegeben in $1 \Omega$ (Ohm)

\subsubsection{Def. d. Widerstands}
Die leicht zu merkende Gleichung U-RI entspricht in ihrer ursprünglichen Form\begin{align} \label{eqn:Definition d. Widerstands}
I 			&= \frac{1}{R}U = G\cdot U
\end{align}\begin{align} \label{eqn:URI}
\boxed{U = R \cdot I}
\end{align}
G steht hier für den Leitwert

\subsection{Einfache Netzwerke von Widerständen}
\subsubsection{Reihenschaltung}
Für Reihenschaltungen von ohmschen Widerständen ergibt sich, dass die jeweiligen Stromstärken überall gleich sind, die Spannungsabfälle über einen Widerstand jedoch von der eigenen Mächtigkeit abhängt. Nach der Maschenregel als Summe über die einzelnen Teilspannungen ergibt sich
\begin{align} \label{eqn:Gesamtwiderstand Reihenschaltung}
\boxed{R_{ges} = \frac{U}{I} = \sum\limits_i R_i}
\end{align}Merke: \textbf{In Reihenschaltungen addieren sich die Widerstände}

\subsubsection{Parallelschaltung}
Parallelschaltungen charakterisieren sich durch eine überall konstant vorliegende Spannung. Einzig die Stromstärke ist unterschiedlich und ermittelt sich durch den jeweiligen Widerstand aus $I_i = \tfrac{U}{R_i}$. Aus der Knotenregel folgt, die Summe aller gleichartigen Ströme ergibt den Gesamtstrom mit
\begin{align} \label{eqn:Gesamtwiderstand Parallelschaltung}
\boxed{R_{ges}= \frac{U}{I_{ges}} = \sum\limits_i \frac{1}{R_i} = \sum\limits_i G_i}
\end{align}Merke: \textbf{Bei Parallelschaltungen addieren sich die Leitwerte}

\subsection{Messgeräte (DMM)}
Um quantitativ unsere aufgestellten Gesetzmäßigkeiten zu überprüfen sollten wir messbare Größen auch mit dafür ausgelegten Messgeräten durchführen. Die Problematik ist ähnlich der Unschärfe im Gebiet der Quantenmechanik, durch jede Messung wird das Ergebnis verfälscht. Wir können dann nicht mehr genau sagen, wie die Welt kurz vor dem Messvorgang sich darstellte.
Können wir diese negativen Auswirkungen technisch verschwindend klein werden lassen, bleibt uns die Möglichkeit  zumindest durch statistische Betrachtungen unsere Ergebnisse und Theorien zu überprüfen.
\paragraph{Amperemeter:} werden \textbf{in Reihe} geschaltet. Wird der Innenwiderstand $R_i$ sehr klein im Verhältnis zum Gesamtwiderstand $R$ der Schaltung, so ergibt sich der Verringerungsfaktor \begin{align*}
\alpha = \frac{R}{R+R_i}
\end{align*}Für $R_i$ spricht man von einem idealen Amperemter. Amperemter müssen immer \textbf{niederohmig} sein
\paragraph{Voltmeter:} werden \textbf{parallel} geschaltet. Ein perfektes Voltmeter besitzt einen unendlich großen Innenwiderstand, so wird die Änderung der fließenden Ladungen kaum beeinträchtig (siehe unbelasteter Spannungsteiler)\begin{align*}
I = \lim\limits_{R \rightarrow \infty}\left( \frac{U}{R}\right)
\end{align*}Voltmeter sollten also einen möglichst großen Innenwiderstand (oft im Bereich von $M\Omega$ )gegen über  der Gesamtschaltung besitzen. Voltmeter müssen \textbf{hochohmig} sein

\subsection{Brückenschaltung: Wheatstone }
