% Einbindung der Präambel
%\documentclass[ngerman]{scrartcl}
\documentclass[%
	pdftex,%              PDFTex verwenden da wir ausschliesslich ein PDF erzeugen.
	a4paper,%             Wir verwenden A4 Papier.
	oneside,%             Einseitiger Druck.
	12pt,              %  Grosse Schrift, besser geeignet f�r A4.
	halfparskip,%         Halbe Zeile Abstand zwischen Abs�tzen.	
	%chapterprefix,%       Kapitel mit 'Kapitel' anschreiben.
	%headsepline,%         Linie nach Kopfzeile.
	footsepline,%         Linie vor Fusszeile.	
	%bibtotocnumbered,%    Literaturverzeichnis im Inhaltsverzeichnis nummeriert einf�gen.
	idxtotoc%             Index ins Inhaltsverzeichnis einfügen.
]{scrartcl}
% Titel, Autor und Datum
% zusäzliche Pakete einbinden
\usepackage[ngerman]{babel}

\usepackage[usenames,dvipsnames]{pstricks}
\usepackage{float}
\usepackage{epsfig}
\usepackage{pstricks}

%erweiterte Tabellen
\usepackage{tabularx}


\usepackage{fancybox}
\usepackage{amsmath}
\usepackage{amsfonts}
\usepackage{amssymb} 
\usepackage{nicefrac}

%erzeugt die Mengen $\mathds{R}^3$
\usepackage{dsfont}
%geschwungene Buchstaben \mathscr{F} 
\usepackage{mathrsfs} 

\usepackage[utf8x]{inputenc}
%\usepackage[T1]{fontenc}
\usepackage{ae}

\usepackage{setspace}
\usepackage[left=25mm,right=25mm,top=28mm,bottom=25mm,headheight=21pt,headsep=27pt,footskip=30pt]{geometry}
\usepackage{blindtext}


\usepackage{graphicx}

%Packet für anklickbare Links
\usepackage{hyperref}

\usepackage[scaled=.85]{luximono}

%\pagestyle{headings}

%
%  4. Stil der Überschriften auf normale Schrift.
%     Wir verwenden für die Überschriften den selben Font wie für den Text.
%
%\setkomafont{chapter}{\Huge}
\setkomafont{section}{\normalfont\bfseries\LARGE}
\setkomafont{title}{\normalfont\Large\bfseries\Large}
\setkomafont{subsection}{\normalfont\bfseries\large}
\setkomafont{subsubsection}{\normalfont\bfseries}
			



\setkomafont{sectioning}{\normalfont\bfseries}       % Titel mit Normalschrift
\setkomafont{captionlabel}{\normalfont\bfseries}     % Fette Beschriftungen 
\setkomafont{pagehead}{\normalfont\itshape}          % Kursive Seitentitel
\setkomafont{descriptionlabel}{\normalfont\bfseries} % Fette Beschreibungstitel


%#uhrzeit
\usepackage{scrtime}

%Kopf- und Fußzeile
\usepackage{fancyhdr}
\pagestyle{fancy}
%\fancyhf{}
 
%Kopfzeile mittig mit Kaptilname
%\fancyhead[C]{\nouppercase{\leftmark}}
%Linie oben
%\renewcommand{\headrulewidth}{0.5pt}
 
%Fußzeile links bzw. innen
\fancyfoot[L]{Erstellt durch Florian Diekmann}
%Fußzeile mittig (Seitennummer)
\fancyfoot[C]{\thepage}
%Linie unten
\renewcommand{\footrulewidth}{0.5pt}
 
% Fußzeile auf jeder Seite - auch Kapitel und Inhaltsverzeichnis
%\fancypagestyle{plain}{%
%   \fancyhf{}%
%   \fancyhead[C]{} %Kapitelname ausblenden
%   \fancyfoot[L]{Text muss hier stehen}
%   \renewcommand{\headrulewidth}{0.0pt} %obere Linie ausblenden
%   \fancyfoot[C]{\thepage}
%}
% anklickbare Links
\definecolor{darkblue}{rgb}{0,0,.5}
\hypersetup{pdftex=true}


%%% eigene Macros
%\newcommand{\1vec}[2]{\mathrel{\hat{\#1}_{\#2}} } %
\newcommand{\entspricht}{\ensuremath{\mathrel{\widehat{=}}}} 
\newcommand{\kreis}[1]{\unitlength1ex\begin{picture}(2.5,2.5)%
\put(0.75,0.75){\circle{2.5}}\put(0.75,0.75){\makebox(0,0){#1}}\end{picture}}
\newcommand{\grad}{\ensuremath{\mathrm{grad\,}}} 
\newcommand{\Fdiv}{\ensuremath{\mathrm{div\,}}} 
\title{Repetitorium - Physik II}
\author{Florian Konstantin Diekmann}
\date{\today}


\begin{document}
\maketitle
\vspace{3cm}
\begin{abstract}
Um den Übergang von der Mechanik und Thermodynamik zum nächsten Kapitel zu vereinfachen sei zunächst die Frage erlaubt: "`Erschließt sich die Elektrodynamik unweigerlich aus der klassischen Mechanik?"' Anders als in der Mechanik, die wir auf handfeste Grundideen ableiten konnten, wir dies zunehmend schwerer fallen. So hat sich die Elektrodynamik als eigenständige Säule der Physik etabliert. 
\end{abstract}
  

Dieser physikalische "`Artikel"' soll keineswegs die Ansprüche an ein Vorlesungsskript erheben. Vielmehr dient es  
dem strukturierten Nacharbeiten des Vorlesungsinhaltes. Wobei immer nur ein kleiner Teil behandelt wird, der mir persönlich für wichtig erscheint. Weiter kann weder   Richtigkeit noch  Vollständigkeit  garantiert werden. Sollten sich auch  inhaltliche bzw. Tippfehler einschleichen, freue ich mich über jede Kontaktaufnahme.
Da ich leider nicht alle Abbildungen selbst gestalten kann, bin ich häufig auf die Abbildungen fremder Lehrbücher angewiesen. So bedeutet ein Vermerk "`Quelle:Demtröder"' eine Abbildung, die aus dem Band 2 der Demtröderreihe entnommen ist.
\newpage
\tableofcontents
\newpage
%
% ############## Inhalt #######################
%
\section{Elektrostatik}
\label{Elektrostatik}Massen erzeugen Gravitationsfelder, heiße Körper ein Temperaturfeld um sic, so erzeugen Ladungen in Analogie ein elektrisches Feld. So soll es hier primär in der Elektrostatik um die Betrachtung statischer elektrischer Ladungen gehen, die miteinander wechselwirken können. Diese Ladungen sind im Gegensatz zu bewegten Ladungen nicht frei beweglich.
\subsection{Elektrische Ladungen}
Es hat sich etabliert den Ladungen zwei Arten zuzuordnen. Damit folgt schon dieses Beispiel nicht mehr der Analogie zum Gravitationsfeld (die sich hauptsächlich  durch eine attraktive Wechselwirkung charakterisiert), indem man zwischen \textbf{negativen} und \textbf{positiven} \textit{Ladungsträgern} unterscheidet. Ladungen bleiben in abgeschlossenen Systemen stets erhalten und sind im Gegensatz zu Massen vom Bewegungszustand unabhängig, Ladungen müssen also vorwiegend nicht relativistisch betrachtet werden. Ausnahme bilden natürlich die Geschwindigkeit von Teilchen etc. \footnote{Dazu mehr in den Versuchen zu beschleunigten Elektronen in Fadenstrahlröhren}. \\
\begin{center}\fbox{Gleichnamig Ladungen stoßen sich ab, ungleichnamige ziehen sich an}\end{center}
%
%
\paragraph{Die Elementarladung} tritt anders als Masse vorwiegend in Quanten auf, also ganzzahligen Vielfachen der Elementarladung e
\begin{align} \label{kons:Elementarladung}
e = 1.602 \cdot 10^{-19} \, C  \quad \mathrm{(Coulomb)}
\end{align}Selbst solche betragsmäßig kleinen Konstanten lassen sich heutzutage sehr genau bestimmen, die geschieht z.B. häufig anhand der Ergebnisse des \textit{Millikan Versuchs}. Überlegungen haben gezeigt, dass schon kleinste Abweichungen um den "`wahren Wert "' Kräfte hervorrufen müssten, die weit größer als die der Gravitationsfelder sein müssten.

\subsection{Kraft zwischen zwei Ladungsträgern}
Die Richtung der Wechselwirkung zwischen zwei Punktladungen erschließt sich aus ihrer Verbindungslinie. Dies scheint zunächst absolut selbstverständlich, warum sollte genau diese Kraft unserem Gefühl widersprechen? Gehen wir davon aus, dass der Raum isotrop ist (von ihm keine Wechselwirkung ausgeht) können wir unsere Vermutung stützen, doch mit solchen Überlegungen muss man stets vorsichtig sein.

Das \textbf{Coulomb-Gesetz} beschreibt die Proportionalität der Kraft zwischen zwei Ladungen die gerade proportional aus dem Produkt beider Ladungen und umgekehrt proportional zum Radiusquadrat ist.
\begin{align} \label{eqn:Coulomb-Gesetz}
\boxed{ \vec{F} = \frac{1}{4\pi \varepsilon_{0}} \frac{q_1 Q_2}{r^2} \hat{r}}
\end{align}Dabei heißt $\varepsilon_{0}$ Dielektrizitätskonstante bzw. Influenzkonstante mit\begin{align*}
\varepsilon_{0} = 8.8542 \cdot 10^{-12} \,\mathrm{ \frac{As}{Vm}}
\end{align*}
\subsection{Das Elektrische Feld}
Wir könnten das Elektrische Feld pragmatisch einführen als $\vec{F} = q\vec{E}$, doch dies ist zugegebener maßen für das weitere Verständnis nicht sehr förderlich. \par
Treten Kräfte auf, obwohl der geladene Körper ruht, so sprechen wir von \textit{Coulomb-Kräften} und beschreiben das Gebiet, in dem diese Kraft auftritt als \textbf{elektrisches Feld}. \\Treten gleichzeitig Kräfte auf, obwohl der Ladungsträger nicht ruht, so sprechen wir hier von \textit{Lorentzkräften}, die zwar proportional zur Geschwindigkeit sind, jedoch vektormäßig senkrecht auf der Geschwindigkeit stehen. Wir sprechen hier von einer Kraftwirkung in einem \textit{Magnetfeld} \par
Erfährt ein Körper in einem elektrischen Feld eine Kraft $\vec{F}$ mit der Ladung Q, so definieren wir die \textbf{elektrische Feldstärke} als \begin{align}
\vec{E} &= \frac{\vec{F}}{Q} 
\end{align}\begin{align} \label{eqn:elFeldstärke}
		\boxed{\vec{E}= \frac{1}{4\pi \varepsilon_{0}} \frac{Q}{r^2} \hat{r}}
\end{align} Nach unserer Definition für die elektrische Feldstärke ergibt sich eine Einheit in SI-Größen von \begin{align}\left[ E\right] = \frac{N}{C} = \frac{V}{m}
\end{align}
\paragraph{} Die für den gesamten $\mathds{R}^3$ definierte elektrische Feldstärke (die z.B. beim Schwerefeld durch $\vec{g}$ bezeichnet wird) ist ein \textbf{Vektorfeld}. Genau diese Eigenschaft der Vektorcharakteristik erschwert uns zunächst eine qualitative Beschreibung. Unter bestimmten Berücksichtigungen können wir jedoch Aussagen treffen. Als Hilfskonstrukt benutzt man so genannte \textbf{Feldlinien}, mit ihnen beschreibt man den Zustand an Äquivalenten Orten. Ihre Anzahl oder Zeichendichte ist willkürlich, kann demnach kein wirkliches Maß für die Feldstärke sein.
\begin{figure}[h]
\includegraphics[scale=0.5]{img/feldlinien_2gleicherLadungen}
\caption{Feldlinienbild für zwei betragsgleiche Punktladungen mit unterschiedlichem Vorzeichen. Demtröder}
\label{pic:Fedlinien-2gegenpolig}
\end{figure}
Solche Fedlinienbilder kennt man zu Genüge aus den meistens physikalischen Lehrbüchern und sollten jedem schon begegnet sein. Dabei vergisst man häufig, dass diese Art der Darstellung nur ein Schnittbild durch das dreidimensionale Feld ist. Damit geht natürlich zwangsläufig die Vergleichbarkeit im gesamten Raum verloren, die wir jedoch durch hoch symmetrische Felder zurückgewinnen können.\\ In Solchen Feldern werden positive Ladungen als Quellen und negative Ladungen als Senken bezeichnet. Elektrische Felder zeigen demnach immer von \textbf{postiven Ladungen hin zu negativen Ladungen}. \begin{quotation}Eine kleine Anekdote sei hier erlaubt und geht auf meinen alten (nun schon pensionierten) zurück. Sollte man einmal die Definition von Quellen und Senken vergessen haben, hilft das $\textbf{n}^4$-Gesetz weiter. Linie\textbf{n} münde\textbf{n} im grüne\textbf{n} Süde\textbf{n}. Und geht damit auf die alte schulische Erklärung von Magnetpolen und den ihn zugeordneten Farben \textit{rot} und \textit{grün} zurück.
\end{quotation}

\subsubsection{Feld einer homogenen Ladungsverteilung} Bisher haben wir uns hauptsächlich mit diskreten Ladungsverteilungen (Punktladungen) beschäftigt. Jede Berechnung von Ladungsverteilungen folgt dem gleichen Prinzip, egal ob langer Leiter oder eine Scheibe. Hier soll exemplarisch das Vorgehen für die Berechnung des Felds für einen unendlich lang ausgedehnten Draht mit der Probeladung $q_1$ durchgeführt werden. \begin{enumerate}
\item Welche Kräfte wirken?
\item Wie groß ist die Ladungsdichte?
\item Integration über "`zerstückelte"' Flächen
\end{enumerate}\begin{figure}\begin{center}
\includegraphics[scale=0.25]{img/homogene_Ladungsverteilung}
\end{center}
\caption{Berechnung einer homogenen Ladungsverteilung eines Leiters}
\label{pic:Ladungsverteilung}
\end{figure}
Zunächst sei gesagt, dass sich der Anteil des elektrischen Feldes stets aus den zwei grundlegenden Richtungen x,y durch vektorielle Addition ableiten lässt. So ergibt sich \begin{align*}
\vec{dE} = \vec{dE}_x + \vec{dE}_y
\end{align*}Schon durch Symmetriebetrachtungen kann man sich überlegen, dass man von jedem Punkt des Leiters eine Linie hin zur Probeladung ziehen kann. Genau diese horizontalen Komponenten heben sich gegenseitig auf, wir können also schreiben
\begin{align*}
\vec{dE} = \vec{dE}_y
\end{align*} Denkt man sich das Vektordreieck, so ist $dE_y = dE \cos \alpha$. Der Betrag des elektrischen Feldes erschließt sich nach (\ref{eqn:elFeldstärke}) mit $r=b$ zu\begin{align}
dE_y = \frac{1}{4\pi \epsilon_{0}} \frac{dQ}{b^2} \cos \alpha
\end{align} Aus der Abbildung \ref{pic:Ladungsverteilung} zeigt sich $b = \frac{a}{\cos \alpha}$ \begin{align*}
dE_y = \frac{1}{4\pi \epsilon_{0}} \frac{dQ}{a^2} \cos^3 \alpha
\end{align*} Für jede Ladungsverteilung gibt es verschiedene Quotienten. Ab folgend nun die Linien, Flächen und Volumenladungsdichte\begin{align}
\lambda &= \frac{dQ}{dL}\\
\sigma 	&=\frac{dQ}{dA}\\
\rho 	&=\frac{dQ}{dV}
\end{align}Im unseren Fall handelt es sich um eine Linienladungsdichte, mit $dL = dx$ und $x = a \tan \alpha$ bzw. $dx = \frac{a}{cos^2 \alpha} d\alpha$\begin{align}
dE_y = \frac{1}{4\pi \epsilon_{0}} \frac{\lambda \cos \alpha}{a} d\alpha
\end{align}Um das gesamte elektrische Feld zu bekommen, integrieren wir über alle Winkel zwischen $-\tfrac{\pi}{2}$ und $\frac{\pi}{2}$\begin{align}
E_y &= \int\limits_{-\tfrac{\pi}{2}}^{\frac{\pi}{2}} \frac{1}{4\pi \epsilon_{0}} \frac{\lambda \cos \alpha}{a} d\alpha \\
E_y &=\frac{\lambda}{2\pi \epsilon_0}\frac{1}{a}
\end{align}
Exakt diese Vorgehensweise lässt sich für jedes hochsymmetrische  elektrische Feld anwenden, so erhält man für einen unendlich ausgedehnten Plattenkondenstaor\begin{align}
E_y = \frac{\sigma}{2\epsilon_0}
\end{align}
\subsubsection{Der elektrische Fluss - Gauß'scher Satz}
\label{sec:elFluss}
Die Idee des elektrischen Fluss erfolgt als Erweiterung des anschaulichen Flussbegriffs aus der Hydrodynamik. Nehmen wir an, dass es endlich viele Feldlinien gäbe, so ordnen wir jeden Feldlinie eine Art Röhre/Schlauch zu. Die gerade so dick ist, dass sich benachbarte gerade nicht berühren können. Man kann sich das folgende Bild sehr schön vorstellen, wenn man es mit dem Wasserfluss vergleicht. Das Medium (elektrisches Feld)  fliest im Röhrengebilde  \textbf{von Quellen hin zu Senken} und wird dann \textbf{elektrischer Fluss} genannt.\\ Das Flüssigkeitsvolumen welches in einer Zeitspanne einen gewissen Querschnitt durchfließt ist demnach der Fluss. Eine alternative Beschreibung ist das Produkt aus der Fließgeschwindigkeit und der durchstrichenen Fläche.\begin{align}
\Psi = AE
\end{align}Diese Definition sind zugegeben sehr umgangssprachlich, erleichtern aber den nächsten Schritt hin zur physikalischen quantitativen Beschreibung. \par 
Steht das elektrische Feld unter einem Winkel $\alpha$ zur Flächennormale der durchströmten Fläche, so ist der äquivalente Teil gegeben durch das Produkt aus Fläche und Feld mal des eingeschlossenen Winkels\begin{align}
\Psi = AE\cos \alpha
\end{align}und erinnert uns stark an das Skalarprodukt zweier vektoriellen Größen. Tatsächlich ergibt sich dieser Zusammenhang als\begin{align}
\Psi = \vec{A} \cdot \vec{E}
\end{align} Wenn sich jedoch entlang der Fläche das elektrische Feld ändert (es z.B. keine Punktladungscharakteristik hat), so verunglückt unsere Definition sofort,können sie retten  jedoch erweitern indem wir über die einzelnen Flächenstücke $d\vec{A}$ integrieren.\begin{align}
\boxed{\Psi = \oint \vec{E} \cdot d\vec{A} = \frac{Q}{\epsilon_0}} \label{eqn:elFluss}
\end{align}

\begin{figure}
\begin{center}\includegraphics[scale=0.6]{img/elektrischerFluss}\end{center}
\caption{Steht das überall konstante E-Feld senkrecht auf der Fläche, so gilt \mbox{$\Psi = AE$}. Steht E(const!) schräg zur Fläche so wirkt nur $\Psi = AE\cos \alpha$. Der allgemeine Fall trifft immer zu, so integrieren wir mit (\ref{eqn:elFluss}). \textit{Abb. S.318 Gerthsen} }
\label{pic:elFluss}
\end{figure}
 \paragraph{Beispiel: elektrischer Fluss einer Punktladung Q.} Hierfür überlegen wir uns zunächst die spezifischen Eigenschaften der umgebenden Sphäre. Die Ladungsverteilung einer Punktladung ist radialsymmetrisch, demnach in Abhängigkeit vom Abstand zum Zentrum schon bestimmt.\begin{align*}
\Psi = \int \limits_{\mathrm{Kugel}} \vec{E} \cdot d\vec{A}
\end{align*}Für das elektrische Feld benutzen wir (\ref{eqn:elFeldstärke}) und erhalten nach einsetzen und ausklammern der Konstanten\begin{align*}
\Psi 	&= \frac{Q}{4\pi\epsilon_0}\int \limits_{\mathrm{Kugel}} \frac{1}{r^2}\hat{r} \cdot \underbrace{d\vec{A}}_{\Rightarrow 4\pi r^2}\\
		&= \frac{Q}{\epsilon_0} \label{eqn:elFluss_Punktladung}
\end{align*}Wir erkennen schnell, dass sich die unschönen Konstanten gegenseitig aufheben. Geschichtlich betrachtet ist genau dies der Grund, warum man den Vorfaktor $\tfrac{1}{4\pi\epsilon_0}$ so gewählt hat. Im alten cgs-system kamen die Coulomb-Gesetze übrigens ohne diese "`unvorteilhaften"' aus, die Kehrseite der  Medaille war die schwierige Integration.

\paragraph{Rekursive Berechnung der Feldstärke:} Nehmen wir an, wir kennen den Fluss eines Felds und sollen Aussagen über die Topologie und das Verhalten des elektrischen Felds machen, so helfen unsere anfänglichen Überlegungen zur Symmetrie enorm. Nehmen wir an, wir kennen $\Psi = \frac{Q}{\epsilon_0}$ und legen um diese Punktladung (wir nehmen zumindest an, dass es sich um solch eine handelt) irgendeine Hüllfäche. Idealisieren diese allerdings gleich zu einer Kugel, das erspart uns eine Menge Rechnerei die sehr komplex werden müsste. \\Die Oberfläche solch einer Kugelsphäre kennen wir mit $4\pi r^2$ und hoffen, dass der Fluss gleichmäßig über die Kugelfläche verteilt auftreten wird. So haben wir für $\Psi$ zwei Angaben.\begin{align*}
\frac{Q}{\epsilon_0} = 4\pi r^2 E
\end{align*}Der Vergleich mit \ref{eqn:elFluss} zeigt uns für die Auflösung nach E\begin{align}
E = \frac{Q}{4\pi \epsilon_0 r^2}
\end{align}Das ist genau die Definition des Betrags des elektrischen Felds einer Punktladung, wie wir es gewohnt sind.
\subsection{Elektrostatisches Potential}
Wenn wir uns an die Mechanik des 1. Semesters erinnern, so fällt im Zusammenhang mit der Potentiellen Energie der Begriff des konservativen Kraftfelds. Dabei stellten wir fest, dass es gerade unwichtig ist welchen Weg wir von Punkt 1 nach 2 wählen. Gleiches gilt auch für das statische elektrische Feld. Die Arbeit W die verrichtet werden muss um die Probeladung Q hin zu einem anderen Punkt im Kraftfeld zu transportieren ist unabhängig vom gewählten Weg. Das elektrische Kraftfeld ist also ein \textbf{konservatives Kraftfeld}. \\ Bewegen wir die Ladung im Feld relativ langsam (in der Größenordnung von 5\%Vakuum-Lichtgeschwindigkeit) so entspricht die Arbeit, die wir aufbringen müssen gerade dem negative Produkt aus $q \vec{E}$ und der Weglänge\begin{align}
dW = \vec{F} \cdot d\vec{s} = - q \vec{E} d\vec{s} \label{eqn:defElArbeit}
\end{align}
So ergibt sich für die Arbeit zwischen zwei Punkten $p_1 \; p_2$\begin{align}
\boxed{ W_{12}	= -q\int \limits_{p_1}^{p_2} \vec{ E} \cdot d\vec{s} }\label{eqn:elArbeit} \end{align}

\begin{align}
0 		= q \oint  \vec{E} \cdot d\vec{s} \Rightarrow \mathrm{konservativ!}
\end{align} Dividiert man die Änderugn der potentiellen Energie durch den Betrag der bewegten Probeladung, erhält man einen Zusammenhang zwischen geleisteter Arbeit von $p_1$ und $p_2$ die wir \textbf{Potentialdifferenz} bzw. häufig auch als \textbf{Spannung} bezeichnen

\subsubsection{Die Spannung}
\begin{align} \label{eqn:defSpannung}
\boxed{U = \frac{W}{q} = \phi\left( p_2\right) - \phi\left( p_1\right) = - \int \limits_{p_1}^{p_2} \vec{ E} \cdot d\vec{s} }
\end{align}Diese Form erscheint auf den ersten etwas ungewohnt, dabei kennen wir sie aus der Schule im Zusammenhang mit Plattenkondensatoren. Die Linearität des Abstands d zweier Platten lässt ein schnelles lösen des Integrals zu:\begin{align*}
E = \frac{U}{d}
\end{align*}
\paragraph*{Beachte:}Von einer Spannung zu sprechen ist relativ sinnfrei, nur durch den Bezug zu einem Punkt im elektrischen Feld kann man von Spannungen sprechen. Besser ist es von einer \textbf{Spannung zwischen zwei Punkten}  zu Sprechen, dies ergibt sich unmittelbar aus der Definition (\ref{eqn:defSpannung})


Soweit zur Arbeit, die auf keinen Fall mit dem Potential verglichen werden darf. \\ Jedem Punkt im elektrischen Feld ordnen wir eine bestimmte Eigenschaft zu, in diesem Fall ein Potentialwert. Wie bei der potentiellen Energie im Schwerefeld der Erde liegt es in unserer Hand den Nullpunkt frei zu bestimmen, es hat sich jedoch als vorteilhaft erwiesen den "`unendlichen Punkt"' als Normierung zu benutzen. So ergibt sich das \textbf{Elektrostatische Potential} als die Aufsummierung der Wegstücke zwischen dem Punkt $p$ hin zum Nullpunkt.\begin{align}
\boxed{\phi \left( p\right) = \int \limits_p^\infty \vec{E} \cdot d\vec{s}} \label{eqn:elPotential}
\end{align}
\paragraph*{Beispiel: Potential im elektrischen Feld einer Punktladung.} Das E-Feld einer Punktladung ergibt sich natürlich als $\vec{E} = \frac{q}{4\pi\epsilon_0} \frac{\hat{r}}{r}$. Potential für E-Felder ergeben sich durch (\ref{eqn:elPotential}). Wir wollen explizit das Potential zwischen p und dem Nullpunkt bestimmen. Da man leider nicht den genauen Weg kennt, wie sich das Teilchen nun ins unendliche bewegen wird, stellt man sich vereinfacht vor, dass wir uns mit dem Teilchen entlang einer Feldlinie R bewegen.\begin{align*}
\phi \left( p\right) 	&= \frac{q}{4\pi\epsilon_0} \int \limits_R^\infty \frac{1}{r^2} dr \\
						&= \frac{q}{4\pi\epsilon_0} \frac{1}{r}\bigg|_R^\infty \\
						&= \frac{q}{4\pi\epsilon_0 R} 
\end{align*} Solch ein Potential hat demnach die Proportionalität von $1/r$ (Abb. \ref{pic:Potential Punktladung})
\begin{figure}[htp]\begin{center}
\includegraphics[scale=0.5]{img/elPotential_Punktladung}\end{center}
\caption{Potential einer Punktladung. Orte gleicher Potentiale werden schwarz dargestellt als Äquipotentialflächen. Zum Vergleich die Feldlinien blau markiert. \textit{Abb. S. 320 Gerthsen}}
\label{pic:Potential Punktladung}
\end{figure}
\subsubsection{Poissongleichung}\begin{align}\label{eqn:Poissongleichung}
\Fdiv \grad \varphi = \boxed{\Delta \varphi = -\frac{\varrho}{\epsilon_0}}
\end{align}
\subsection{Kondensator}
Ein Kondensator ist ein Ladungsspeicher für elektrostatische Ladungsträger.
Durch externe Aufladung kann sich ein Potential aufbauen, welches wir Spannung nennen.
Diese Spannung ergibt mit  Plattenabstand d und einer homogenen Ladung beingt durch (\ref{eqn:elFluss})\begin{align}
E = \frac{U}{d}
\end{align}Die Spannung ist ein Maß für die gespeicherte Ladung C. Da $U \propto Q$ gilt, findet man die Proportionalitätskonstante \textbf{C} als die \textbf{Kapazität des Kondensators}.

\subsubsection{Kapazität eines Plattenkondensators}
Für einen metallischen Kugelkondensator (manchmal auch als Kugelkonduktor bezeichnet, wenn sich die äußere Gauß-Schale im Unendlichen befindet) \footnote{Einfacher ist es jedoch zunächst den Begriff des Kugelkondensators zu benutzen und zwischen Schalenkugelkondensatoren zu unterschieden} mit dem Radius R ergibt sich durch das Potential $\phi \left(R\right)$(\ref{eqn:elPotential}), und anschließendem Umformen nach der gespeicherten Ladungsmenge: $Q = 4\pi \epsilon_0 R \cdot U$. Wobei wir hier für die Spannung U das Nullpotential  mit $r = \infty$ entsprechend der üblichen Konvention gewählt haben (Wir schreiben also die Potentialdifferenz nicht explizit auf, sollten sie aber trotzdem erwähnen).
Die beschriebene Poportionalitätskonstante ist hierbei $C_{Kugel} = 4\pi \epsilon_0 R$. \par
Für eine weitere Abstrahierung kann man den kapazitiven Ausdruck betrachten. $4\pi R$ ist gerade die Oberfläche A der Kugel dividiert durch den entsprechenden Radius der Kugel. Schreiben wir also $\tfrac{Q}{U} = \epsilon_0 \frac{A}{R} $\begin{align} \label{eqn:Kapazität Plattenkondensator}
\boxed{C = \frac{Q}{U} = \epsilon_0 \frac{A}{R}}
\end{align}
Die Einheit der Kapazität ist zu Ehren Faradays durch den Namenszusatz "`\textbf{Farad} "' bezeichnet.\begin{align*}
\left[ C\right] = 1 \frac{As}{V} = 1 \, F
\end{align*}

Hinweis: Um Kapazitäten von Symmetrischen Objekten zu berechnen kann immer der Umweg über das Potential benutzt werden. Voraussetzung ist natürlich, dass man das jeweilige spezifische E-Feld kennt. Hier einige \textbf{Beispiele für Kapazitäten} von Kugelschalenkondensator, Zylinderkondensator,Plattenkondensator sowie Paralleldraht
\begin{align*}
C_{KS} 		&= 4\pi \epsilon_0 \frac{r_1 r_2}{r_2 - r_1} \\
C_{Zyl} 	&= \frac{2\pi \epsilon_0 h}{\ln  \left(r_2 / r_1\right) }\\
C_{Platt} 	&= \epsilon_0 \frac{A}{d}\\
C_{ParaDraht} &=  \frac{\pi \epsilon_0}{\ln\left( d/r\right)}
\end{align*}
Beachte: Die vorletzte Formel wurde schon im Beispiel (\ref{eqn:Kapazität Plattenkondensator}) behandelt. Sie ist auch für alle anderen Körper gültig, so lange der Abstand d hinreichend klein ist.

\paragraph{Parallelschaltung von Kondensatoren}
\begin{align} \label{eqn:gesKapazität Parallelschaltung}
\boxed{C_{ges} = \sum \limits_{i=1}^{N} C_i}
\end{align}

\paragraph{Serienschaltung von Kondensatoren}
\begin{align} \label{eqn:gesKapazität Serienschaltung}
\boxed{\frac{1}{C_{ges}} = \sum \limits_{i=1}^{N} \frac{1}{C_i}}
\end{align}

\subsection{Energie im elektrischen Feld}
Stellen wir uns einen geladenen Plattenkondensator vor, dem wir eine kleine Ladung $dQ$ hinzufügen möchten. Im Fall, dass der Kondensator schon vorgeladen ist, existiert ein E-Feld. Gegen gerade dieses E-Feld muss Arbeit verrichtet werden (Der Millikan-Versuch macht sich diese Kraftwirkung zu nutze und erzeugt wunderbare Schwebeffekte). Es  gilt für die Kraftwirkung 

\begin{align*}
d\vec{F} = \vec{E} \cdot dQ
\end{align*}
Für die Arbeit gilt nach der klassischen Mechanik $ W = \int \vec{F} \cdot d\vec{r}$. Setzen wir also dies um \begin{align*}
dW = - dQ \int \limits_0^d  \vec{E} \cdot d\vec{s}
\end{align*} Der Teil nach $dQ$ entspricht gerade dem Potential zwischen Anfangs und Endpunkt (\ref{eqn:defSpannung})\begin{align*}
dW = dQ \cdot U
\end{align*}Bisher haben wir lediglich gezeigt wie sich die Arbeit für den kleinst möglichen Ladungsteil verhält. Um jedoch die Energie zu bestimmen, die in einem aufgeladenen Kondensator steckt, betrachten wir jetzt alle Ladungsträger die wir hin zu anderen Platte bewegen möchten. Die Integration liefert uns \begin{align*}
W 	&= \int \limits_0^Q \frac{Q}{C}\,dQ = \frac{1}{2}\frac{1}{C}Q^2 
\end{align*}Im alltäglichen Gebruach hat es sich als praktisch erwiesen i
sich je nach Gegebenheit $\frac{Q^2}{C} $ durch $Q = C\cdot U$ entsprechend abzuändern. \par
Ist die Ladungsverschiebung abgeschlossen, so ist die gesamte geleistete Arbeit übergegangen in eine Form, die wir \textbf{potentielle Energie} des Kondensators nennen.
\begin{align} \label{eqn:elektrische Energie}
\boxed{E_{pot} = \tfrac{1}{2} C U^2 = \tfrac{1}{2}QU}
\end{align}


\subsubsection{Energiedichte von elektrischen Feldern}
Nachdem wir den Kondensator erfolgreich aufladen konnten und ihm eine charakteristische Energie zuordnen konnten darf die Frage erlaubt sein wo sich denn diese Energie nun befindet. \\ Da sie weder außen an den Platten noch in selbiger zu finden sind, liegt es nahe, dass die \textbf{Energie im inneren des Kondensators} beheimatet sein sollte. durch die Aufladung des Kondensators erzeugen wir ein E-Feld, wir vermuten daher, dass genau hier die Energie gespeichert wird. \par 
Äquivalent zur Dichte bei Massen berechnet sich die Energiedichte aus  dem Quotienten der gespeicherten \textbf{Energie pro Volumen}.\begin{align*}
w = \frac{E_{pot}}{V}
\end{align*} Für den speziellen Fall eines \textbf{Plattenkondensators} ergibt sich weiter mit $V = A \cdot d$ ( wobei $d$ den Abstand beider Platten zueinander beschreibt) und der Kapazität $C = \epsilon_0 \frac{A}{d}$ Sowie der Feldstärle $E = \frac{U}{d}$
\begin{align*}
w = \frac{\tfrac{1}{2} C \cdot U^2}{a \cdot d} = \frac{\epsilon_0 \cdot A \cdot E^2 \cdot d^2}{2 \cdot A \cdot d^2 }
\end{align*}

Weitere Beispielrechnungen  würden zeigen, dass für jede Art von elektrischen Feldern die \textbf{elektrische Energiedichte} stets gegeben ist durch
\begin{align} \label{eqn:Energiedichte}
\boxed{w = \tfrac{1}{2} \epsilon_0 |\vec{E}|^2}
\end{align}


\subsection{Dielektrika}

Zugegeben sind Dielektrika recht seltsame Objekte,auf eine explizite Herleitung soll daher verzichtet werden (ein paar wichtige Beziehungen habe ich dennoch beigefügt). Wichtiger ist es die Phänomene die so ein besonderer Isolator erzeugt  zu verstehen und deuten zu können. 
Nehmen wir uns einen geladenen Plattenkondenstor, in selbigen schieben wir ein Dielektrikum, der primär als Isolator wirkt.
In diesem Isolator selbst sind eine große Menge neuer Ladungsträger enthalten, die sich entsprechend dem Dipolmoment genau entgegen der Polarisierung des Plattenkondensators ausrichten (wie die Kompassnadel im Magnetfeld der Erde).
Für die Kapazität des Ladungsträgers ergibt sich jetzt eine \textbf{wesentlich höhere Kapazität}, da wir sehr viel mehr Ladungsträger polarisieren können.
Diese Ladungserhöhung ist je nach gewählten Stoff sehr unterschiedlich.
Diese Materialkonstante fließt direkt in die Berechnung der Kapazität ein. So ergibt sich jetzt wür einen Plattenkondensator mit \textbf{Dielektrizitätszahl} $\epsilon_r$
\begin{align}\label{eqn:Dielektrikum Plattenkondensator}
C = \underbrace{\epsilon_0 \epsilon_r}_{\epsilon} \frac{A}{d}
\end{align}

Diese spezifische Eigenschaft der Dielektrika nutzt man in der technischen Anwendung für verschiedenste Bereiche. Drei wichtige Anwendungen sind\begin{enumerate}
\item \textbf{Geringe Schichtabstände} lassen sich so realisieren, \textbf{ohne die Gefahr von Kurzschlüssen}.
\item Luft besitzt im allgemeinen lediglich eine \textbf{Spannungsfestigkeit} von $\approx 3\mathrm{\tfrac{kV}{mm}}$, hingegen ein Dielektrium sogar $\approx 60\mathrm{\tfrac{kV}{mm}}$ erreichen kann. So kann  sehr hohe Energiedichten erreichen, bei gleichbleibend kleinen Ausmaßen.
\item  Durch die \textbf{höhere Kapazität} können kleinere Kondensatoren gebaut werden, die zudem in der Produktion günstiger geraten. 
\end{enumerate} \par

\textit{\quote  "` Warum kamen wir bisher ohne Dielektrika aus ?"'}
Diese Frage ist leicht zu beantworten. Denn das Dielektrikum, welches wir für unsere bisherigen Rechnungen benutzt haben (ohne es genauer gewusst zu haben) ist das Medium Luft. Der glückliche Zufall ist jedoch, dass dessen Materialkonstante $\epsilon_r$ gerade ungefähr durch 1  
\subsubsection{Dielektrische Polarisation}
\begin{align} \label{eqn:Polarisation}
\vec{p} = q \vec{d}
\end{align}

\subsubsection{Polarisationsladungen}
\begin{align} \label{eqn:Polarisationsladung}
\sigma_{pol}= \frac{Q_{pol}}{A} = \frac{N \cdot q \cdot d \cdot A}{A} = P
\end{align}

	\paragraph{Polarisation}\begin{align} \label{eqn:Polarisation Überlagerung}
\vec{E}_{Diel} = \frac{\sigma_{frei} - \sigma_{pol}}{\epsilon_0} = \vec{E}_{Vak} - \frac{\vec{P}}{\epsilon_0}
\end{align}

\paragraph{Dielektrische Suzeptibilität}
\begin{align} \label{eqn:Dielektrika Feldstärke}
\boxed{\vec{E}_{Diel} = \frac{1}{\epsilon} \vec{E_{Vak}} = \frac{1}{1+ \chi} \vec{E}_{Vak}}
\end{align}

\newpage
\section{Der elektrische Strom}
In der Elektrostatik beschäftigten wir uns zunächst mit statisch ruhenden Teilchen. Die besondere Eigenschaft war hier gerade, dass längs eines Leiters keine Potentialdifferenz auftrat.
\par Jede Bewegung von elektrischen Ladungen bezeichnen wir als einen \textbf{elektrischen Strom}, dieser ist definiert durch die Bewegung positiver Ladungsträger.
Wie kommt es jedoch zu diesem \textbf{Teilchentransport}, der die Quelle jeder Teilchen sein sollte? In den meisten Fällen handelt es sich hier um externe Spannungsquellen, welche je nach Polung einen Elektronenmangel bzw. Elektronenüberschuss erzeugen können.
Diese Elektromotorische Kraft (EMK) hält die Bewegung der Teilchen aufrecht. Ohne diese externe \textbf{Potentialdifferenz }fließen die Elektronen nach einiger Zeit nicht mehr und ruhen. Genau dann befinden wir uns im elektrostatischen Zustand.

\subsection{Stromstärke und Stromtransport}

\paragraph{Elektrischer Strom}
Ein elektrischer Strom beschreibt die Anzahl an Ladungen $dQ$ , die pro Zeitintervall $dt$ einen Leiterquerschnitt durchfließen. Allgemein ergibt sich
\begin{align} \label{def:Strom}
\boxed{I := \frac{dQ}{dt}}
\end{align}
Die Einheit der Stromstärke ist festgelegt durch $1\mathrm{A(mpere)}$. Ampere ist eine Basisgröße im SI-System lässt sich folglich nicht durch andere Größen darstellen. 

Für Gleichströme können wir uns überlegen, dass die Ladungsmenge endlich gegeben ist
 mit $Q = n\cdot q$t (Ladung kann schließlich nur in Quanten auftreten). Es folgte für das Zeitintervall $T$ 
 \begin{align} \label{eqn:elektrischer Strom Gleichspannung}
I_{gl} = \frac{n\cdot q}{T}
\end{align}

\subsection{Kirchhoffsche Regeln}
Netzwerke von Schaltungen können unterschiedlich aufgebaut sein. Man unterscheidet zwischen so genannten Knoten- und Maschentopologien. 
\subsubsection{Knotenregeln}
In Knoten kann keine Ladung gespeichert werden, so sollte die \textbf{Summe der zufließenden Ströme den abfließenden }entsprechen. 
\begin{align} \label{eqn:Knotenregel}
\boxed{\sum \limits_{k=1}^n I_k = 0}
\end{align}

\subsubsection{Maschenregel}
Ein zweiter möglicher Aufbau ist eine Masche. Umläuft man einmal eine gesamte Masche, so muss die \textbf{Summe aller Potentialdifferenzen null} ergeben. Hierbei ist es unabhängig, wie der Umlaufsinn festgelegt wurde.
Die Wichtung von Klemmspannung und den Spannungsabfällen an Widerständen (etc.) erfolgt anti-parallel, also mit unterschiedlichem Vorzeichen. Die Kirchhoffschen Regeln kann man sich bildlich als Vektoraddition hin zum Nullvektor vorstellen.

\begin{align} \label{eqn:Maschenregel}
\boxed{\sum \limits_{k=1}^n U_k = 0}
\end{align}

\subsection{Stromdichte}
Stromdichte $\vec{j}$ ist eine vektorielle Größe und beschreibt den Ladungstransport durch einen Leiter mit einem bestimmten Querschnitt A. Für gutartige Leiter gilt demnach
 \begin{align*}
j = \frac{I}{A}
\end{align*}
Besser ist jedoch die vektorielle Aufsummierung über die Flächennormalen von Querschnittsflächen. 
\begin{align} 
\boxed{I = \int \limits_A \vec{j} \cdot d\vec{A}}
\end{align}
Hauptträger der elektrischen Ladung in el. Leitern sind vorwiegend Elektronen, weitere Vertreter sind Ionen die z.B: Ladungsaustausch in Ionengittern (Elektrolyten etc.) ermöglichen. In Plasmen treten nicht nur spannende Effekte wie z.B. die Polarlichter auf, hier werden zugleich Ionen und Elektronen für den Austausch von Ladungen benötigt.

% Diese Def. ist nicht mehr zeitgemäß!
\begin{align} \label{eqn:Stromdichte}\begin{split}
\vec{j} &= nq\vec{v} \\
\vec{j}	&= \varrho_{el}\vec{v}
\end{split}
\end{align}
Mit der Ladungsdichte geschrieben als $\varrho_{el} = nq$ (bemerke: Die Ladungsdichte nicht mit Dichten von Massenverteilungen bzw. dem elektrischen Widerstand verwechseln!)

\subsection{Ohmsche Widerstände}
Durchlaufen Elektronen einen ohmschen Leiter, so finden fortlaufend Stoßprozesse statt, die zur Verlangsamung selbiger führen. Gitterschwingungen im Leiter selber sind die überwiegenden Effekte, die zu Reibungseffekten führen können.

\subsubsection{Ohmsches Gesetz}
Viele dürften mit dem Begriff des Ohmschen Gesetzes die Gleichung $U = R\cdot I$ verbinden. Leider ist dies ein großer Irrtum, dass Ohmsche Gesetz beschreibt die Proportionalität von homogenen ohmschen Widerständen zur Leiterlänge und die Proportionalität zum Kehrwert des Leiterquerschnitts.
\begin{align} \label{eqn:Ohmsches Gesetz}
\boxed{R = \varrho_{el} \frac{l}{A}}
\end{align}Wobei $\varrho_{el}$ den spezifischen Widerstand beschreibt ($\frac{1}{\varrho_{el}} =  \sigma$) Die Einheit von ohmschen Widerständen wir angegeben in $1 \Omega$ (Ohm)

\subsubsection{Def. d. Widerstands}
Die leicht zu merkende Gleichung U-RI entspricht in ihrer ursprünglichen Form\begin{align} \label{eqn:Definition d. Widerstands}
I 			&= \frac{1}{R}U = G\cdot U
\end{align}\begin{align} \label{eqn:URI}
\boxed{U = R \cdot I}
\end{align}
G steht hier für den Leitwert

\subsection{Einfache Netzwerke von Widerständen}
\subsubsection{Reihenschaltung}
Für Reihenschaltungen von ohmschen Widerständen ergibt sich, dass die jeweiligen Stromstärken überall gleich sind, die Spannungsabfälle über einen Widerstand jedoch von der eigenen Mächtigkeit abhängt. Nach der Maschenregel als Summe über die einzelnen Teilspannungen ergibt sich
\begin{align} \label{eqn:Gesamtwiderstand Reihenschaltung}
\boxed{R_{ges} = \frac{U}{I} = \sum\limits_i R_i}
\end{align}Merke: \textbf{In Reihenschaltungen addieren sich die Widerstände}

\subsubsection{Parallelschaltung}
Parallelschaltungen charakterisieren sich durch eine überall konstant vorliegende Spannung. Einzig die Stromstärke ist unterschiedlich und ermittelt sich durch den jeweiligen Widerstand aus $I_i = \tfrac{U}{R_i}$. Aus der Knotenregel folgt, die Summe aller gleichartigen Ströme ergibt den Gesamtstrom mit
\begin{align} \label{eqn:Gesamtwiderstand Parallelschaltung}
\boxed{R_{ges}= \frac{U}{I_{ges}} = \sum\limits_i \frac{1}{R_i} = \sum\limits_i G_i}
\end{align}Merke: \textbf{Bei Parallelschaltungen addieren sich die Leitwerte}

\subsection{Messgeräte (DMM)}
Um quantitativ unsere aufgestellten Gesetzmäßigkeiten zu überprüfen sollten wir messbare Größen auch mit dafür ausgelegten Messgeräten durchführen. Die Problematik ist ähnlich der Unschärfe im Gebiet der Quantenmechanik, durch jede Messung wird das Ergebnis verfälscht. Wir können dann nicht mehr genau sagen, wie die Welt kurz vor dem Messvorgang sich darstellte.
Können wir diese negativen Auswirkungen technisch verschwindend klein werden lassen, bleibt uns die Möglichkeit  zumindest durch statistische Betrachtungen unsere Ergebnisse und Theorien zu überprüfen.
\paragraph{Amperemeter:} werden \textbf{in Reihe} geschaltet. Wird der Innenwiderstand $R_i$ sehr klein im Verhältnis zum Gesamtwiderstand $R$ der Schaltung, so ergibt sich der Verringerungsfaktor \begin{align*}
\alpha = \frac{R}{R+R_i}
\end{align*}Für $R_i$ spricht man von einem idealen Amperemter. Amperemter müssen immer \textbf{niederohmig} sein
\paragraph{Voltmeter:} werden \textbf{parallel} geschaltet. Ein perfektes Voltmeter besitzt einen unendlich großen Innenwiderstand, so wird die Änderung der fließenden Ladungen kaum beeinträchtig (siehe unbelasteter Spannungsteiler)\begin{align*}
I = \lim\limits_{R \rightarrow \infty}\left( \frac{U}{R} \right)
\end{align*}Voltmeter sollten also einen möglichst großen Innenwiderstand (oft im Bereich von $M\Omega$ )gegen über  der Gesamtschaltung besitzen. Voltmeter müssen \textbf{hochohmig} sein

\subsection{Brückenschaltung: Wheatstone}
\begin{figure}\begin{center}
\includegraphics[scale=0.3]{img/Wheatstone}\end{center}
\caption{Wheatstone Brückenschaltung zur Messung vom $R_x$}
\label{pic:Wheatstone}
\end{figure}
Die Wheatstone Brückenschaltung (Abb. \ref{pic:Wheatstone}) dient zur Bestimmung von zunächst unbekannten Widerständen. Die Schaltung ist sternförmig mit 4 Widerständen aufgebaut, wobei je 2 in Reihe geschaltet und zusammen eine Parallelschaltung darstellen.
In einer dieser Schaltung ist der unbekannte Widerstand. Wichtig ist, dass mindestens einer der Widerstände veränderbar.
Diesen Widerstand verändern wir in seiner Stärke so lange, bis durch die aufgespannte Brücke kein Strom mehr fliest, man spricht von einem \textbf{Abgleich der Brücke}.
So sind demnach die Spannungsabfälle an $R_x$ und $R_3$ gleich (bzw. die gleiche Menge Ladung pro Zeit fließt dort entlang). Es ist damit
\begin{align*} \begin{split}
I_1 R_1 = I_2 R_2 \\
I_1 R_x = I_2 R_3\end{split}
\end{align*}Wir  teilen nun die obere durch die untere Gleichung und erhalten die bekannte Wheatstone Gleichung:
\begin{align} \label{eqn:Wheatstone Schaltung}
\boxed{\frac{R_1}{R_x} = \frac{R_2}{R_x}}
\end{align}

\subsection{Energie und Leistung elektrischer Ströme}
Verschiebt sich eine Ladung Q (egal ob nun durch externe Krafteinwirkung oder durch zauberhafte Influenz) von Punkt P1 zu P2 so wird Energie in Form von 
\begin{align} \label{eqn:Energie el. Strom}
E_{frei} = Q\cdot U
\end{align}zugeführt. Erinnerung: Diese Beziehung hatten wir durch das elektrostatische Potential für radialsymmetrische Felder gezeigt (\ref{eqn:defSpannung})
Gemäß der Energie in einem Kondensator kann Energie im E-Feld gespeichert werden. Dies geschieht z.B. über einer Ladung und dessen kinetische Energie.
\paragraph{Beispiel: Rasende Elektronen} \label{bsp:Elektronenbeschleunigung}
Für viele Anwendungen benötigt man schnelle Elektronen die eine Bewegungsenergie besitzen (z.B. für Kollisionsuntersuchungen oder der \textbf{Fadenstrahlröhre}). Erhitzen wir eine Glühwendel durch Strom, so wird diese ab einer bestimmten Stromstärke im sichtbaren Spektrum leuchten. Genau hier treten Elektronen aus dem Leiter, man nennt dies den "`glühelektrischen Effekt"'.
Diese erzeugten Elektronen sollen in einem E-Feld beschleunigt werden (Die geschieht z.B. durch eine positiv geladene Metallplatte, zu dieser die Elektronen hin wandern möchten). Gehen wir davon aus, dass die \textbf{komplette elektrische Energie in Bewegungsenergie} umgewandelt werden kann ergibt sich für die Beträge sich\begin{align*}
E = q \cdot U &= \frac{1}{2} m v^2 \\
 v &= \sqrt{\frac{2qU}{m}} 
\end{align*}Die Beschleunigung kann so aber nur störungsfrei ablaufen, wenn es nicht zu viele Stoßpartner gibt. Ideal sit natürlich ein Fein-Vakuum. Wobei diese schöne Formel zudem nur für geringe Geschwindigkeiten gilt, bis hin zu ungefähr 5\% Vakuumlichtgeschwindigkeit.

Was ist uns also gelungen? Wir können Teilchen Energien zuordnen, obwohl diese selbst keine Ladungsmaterie sind, sie sind nur die Überträger. Es zeigt sich als sinnvoll eine Basisgröße einzuführen. \textbf{Elektronenvolt} beschreibt die  Energie, die ein Elektron mit der Ladung e und der Spannung 1V erfährt. \begin{align} \label{eqn:Elektronenvolt}
\boxed{1eV = 1.6 \cdot 10^{-19} C \cdot V = 1.6 \cdot 10^{-19}J}
\end{align}
 
 Die Vorstellung, dass Elektronen Energie Energie transportieren können ist heikel, denn schon in Leitern wird diese Energie meist in thermische Wärmestrahlung umgewandelt. Die Größenordnung dieser Abstrahlung ergibt sich nach dem Gesetz von Joule
 
 
\subsection{Gesetz von Joule - el. Leistung}
Erinnern wir uns an die Arbeit W (\ref{eqn:Energie el. Strom}). Für U=const ergibt sich die zeitliche Änderung der Arbeit als Leistung (alles strikt nach der klassischen Mechanik)
\begin{align} \label{eqn:elLeistung}
P = \dot{W} = \frac{d (U\cdot Q)}{dt} = U \cdot \dot{Q} = U \cdot I
\end{align} Merken wir uns die einfache Form für die \textbf{elektrische Leistung}
\begin{align*}
\boxed{P = U \cdot I}
\end{align*}
Für einen ohmschen Leiter ergibt sich mit $U = RI$, dass $P = RI^2$ ist.
\newpage
\section{Magnetostatik}
Ähnlich den elektrischen Feldern wollen wir nun das Phänomen von magnetischen Feldern beschreiben.
Vorweg soll nicht unerwähnt bleiben, dass wir dass "`Warum"' der Magnetostatik nicht erklären können. Dies ist der Quantenphysik vorbehalten, bleibt uns also zunächst verschlossen.
Sehr wohl können wir die Ergebnisse, die die Magnetostatik erzeugt genauer untersuchen. Ladungen in solchen Feldern, Ablenkungen und Feldlinien. Vieles, was wir noch aus der Elektrostatik kennen, finden wir hier wieder.

\subsection{Die  magnetische Feldstärke}
Magnetische Felder werden durch den Begriff der Feldstärke charakterisiert. Eine übliche Abkürzung hierfür sind \textbf{B-Felder}. Die Einheit dieser Größe wird beschrieben durch \textit{Tesla}. \begin{align} \label{eqn:EinheitTesla}
\left[B \right] = 1T
\end{align}
\paragraph{Feldlinienbilder} konzentrisch um langen Leiter, horizontae lange Linien entlang eines Leiters von oben betrachtet. ACHTUNG Bilder ergänzen!

\paragraph{Feldstärke} Zwei gerade parallel angeordnete Leiterströme (mit anti-parallel verlaufenden Strömen) erfahren eine Abstoßung. \\
Wir nehmen zunächst an, dass jeder einzelne \textbf{Leiter} ein\textbf{ B-Feld erzeugen} kann. Wobei jedes Feld senkrecht zum Leiter und radial mit Abstand zum Leiter verläuft. 

Die Versuche von Oersted und Rowland zeigten, dass diese Vorstellung keineswegs abstrus ist. So konnten sie zeigen, dass Magnetfelder von stromdurchflossenen Leitern stets proportional zum Quotienten aus Stromstärke I und Abstand  r sind.
\begin{align*}
B \propto \frac{I}{r}
\end{align*}
Mit der  Proportionalitätskonstante $\frac{\mu_0}{2\pi}$ können wir unserer erste Gleichung aufstellen
\begin{align} \label{eqn:magFeldstärke}
B_i = \frac{\mu_0}{2\pi} \frac{I_i}{r}
\end{align}
Für den allgemeinen vektoriellen Fall ergibt sich das die \textbf{Feldstärke eines unendlich lang ausgedehnten Leiter} und dem konstanten Strom I
\begin{align} \label{def:magFeldstärke}
\boxed{\vec{B} = \frac{\mu_0}{2\pi} \frac{I}{r} \hat{r}}
\end{align}
\subsection{Kraft zwischen zwei stromdurchflossenen Leitern}
Für einen Leiter der Länge L und der Stromstärke I ergibt sich für die Kraft eines weiteren stromdurchflossenen Leiters (Voraussetzung: Die Größen $l,B,I$ bilden eine orthogonale Basis).
\begin{align} \label{eqn:KraftstromDruchfl}
F = I \cdot l \cdot B
\end{align} Das Produkt $I \cdot l$ kann als die "`Ladung"' angesehen werden, entspricht damit der Vorstellung einer Kraft auf  Probeladungen im elektrischen Feld (\ref{eqn:Coulomb-Gesetz}).
Man fragt sich jetzt was passiert, wenn die Größen nicht senkrecht aufeinander stehen. Die Amperschen Versuche zeigten, dass Leitströme senkrecht zueinander keine Kraftwirkung hervorrufen.
Es wirkt also jeweils nur die jeweilige senkrechte Komponente $\sin \left(l,B \right)$. Dieses Verhalten erinnert uns an das Vektorprodukt zweier Größen.
\\ Und tatsächlich gilt für die Kraft im vektoriellen Fall
\begin{align} \label{eqn:KRaftBFeld_vec}
\boxed{\vec{F} = I \left( \vec{l} \times \vec{B}\right)}
\end{align}
Bedingung ist jedoch, dass über die komplette Leiterlänge das B-Feld konstant bleibt. Ist dies nicht gegeben, so ist es sinnvoll genügend kleine Leiterstückchen zu betrachten um inhomogene Betrachtungen durch Integration zu vereinfachen. Es ergibt sich
\begin{align} \label{eqn:KraftBFeld_infintesimal}
d\vec{F} = I \left( d\vec{l} \times \vec{B}\right)
\end{align}
\subsection{Die Lorentzkraft}
Erlauben wir uns zunächst eine kleine Vorbemerkung. Elektrischer Stromfluss kann nur in einem geschlossenen Leiter fließen.
\begin{align*}
\oint \limits_{Stromkreis} I d\vec{l} = 0
\end{align*} Solch ein Stromfeld muss also konservativ sein, Ladungen dürfen nicht verloren gehen. Diese Eigenschaft wird uns für viele Überlegungen nützlich sein, auch wenn sie manchmal nicht explizit erwähnt wird.

Bewegen sich freie Elektronen (dQ) im Leiter mit einer festen Geschwindigkeit v und durchstreichen dabei ein Leiterstück dl
\begin{align*}
\vec{v }			&= \frac{d\vec{l}}{dt}\\
dQ \cdot \vec{v} 	&= dQ  \cdot \frac{dl}{dt} \\
			&= I d\vec{l}
\end{align*}
Setzen wir den letzten Ausdruck in Gleichung \ref{eqn:KraftBFeld_infintesimal} ein, so ergibt sich 
\begin{align*}
d\vec{F} = dQ \left( \vec{v} \times \vec{B} \right) \qquad \bigg| \int
\end{align*} Die sich ergebende Kraft nennen wir \textbf{Lorentzkraft}
\begin{align} \label{eqn:Lorentzkraft}
\vec{F} = q \left( \vec{v} \times \vec{B} \right)
\end{align}
Wirkt zudem ein elektrisches Feld auf den Ladungsträger, überlagert sich die Kraftwirkung von E-Feld und B-Feld
\begin{align} \label{eqn:Lorentzkraft E-Feld}
\vec{F} = q \left( \vec{E} + \vec{v}  \times \vec{B}\right)
\end{align}
\subsection{Fadenstrahlröhre}
Ein typischer Einführungsversuch in die Atomphysik, dieser ist jedoch so anschaulich für die Lorentzkraft, so dass wir ihn hier genauer behandeln wollen.

\begin{figure}\begin{center}
\includegraphics[scale=0.5]{img/fadenstrahl}\end{center}
\caption{Typischer Aufbau einer Fadenstrahlröhre. Bildquelle \textit{LP Uni Göttingen}}
\label{pic:Fadenstrahlröhre}
\end{figure}

\subparagraph{Versuchsanordnung} Eine Glühwendel erzeugt durch den glühelektrischen Effekt freie Elektronen (Elektronenwolke). Die werden durch eine Anoden Beschleunigungsstrecke geführt. Eine Fokussierung bewirkt der Wehneltzylinder (Geschwindigkeitsfilter).
Diese schnellen Elektronen werden in einem Glaszylinder mit Leuchtgas geführt (Druck zwischen 0.1 - 1Bar). Je nach angelegtem homogen Magnetfeld werden die Elektronen unterschiedlich abgelenkt. 
\subparagraph{Beobachtung}Treffen schnelle Elektronen in das Gasgemisch, so kann man sehen, dass sich ein kleiner Lichtstrahl ausbildet.
Dieser Lichtstrahl  bildet sich tangential zu den ausströmenden Elektronen. Wird ein homogenes Magnetfeld angelegt (erzeugt durch ein Helmholzspulenpaar des Radius R im Abstand R) so werden die Elektronen abgelenkt.
Je nach Ausströmrichtung der Elektronen bildet sich eine Kreisbahn oder aber eine Schraubenbahn aus. 
\subparagraph{Wichtige Gleichungen}
Die Elektronen werden aus dem Ruhezustand ($v_a \approx 0$) durch die Anode im E-Feld beschleunigt. Wird diese komplett in innere kinetische Energie umgewandelt, welche wir gerade durch die Def. der Spannung eingeführt hatten (\ref{eqn:defSpannung}) $W = e  U $.
\begin{align*}
\boxed{e \cdot U = \frac{1}{2}m \vec{v}^{\, 2}}
\end{align*} Betrachten wir ab jetzt nur noch den Fall, dass alle wichtigen Größen eine Basis bilden. \par 
\begin{align} \label{eqn:ElektronenFadenstrahlGeschw}
v = \sqrt{\frac{2 \cdot U \cdot e}{m}}
\end{align}Mit e als Elementarladung, m die Elektronenmasse und U der Beschleunigungsspannung

Diese schöne Gleichung gibt uns nicht nur die Möglichkeit die\textbf{ spezifische Ladung} $\tfrac{e}{m}$ zu bestimmen, wir können so auch näherungsweise die mittlere \textbf{Geschwindigkeit} ermitteln, mit der die Elektronen das Leuchtgas anregen, worauf dieses Licht emittiert (mehr dazu in Physik 3). Schon für Spannungen von einigen kV ergeben sich Geschwindigkeiten von $2 \cdot 10^{7} m/s$ , schon höhere Spannungen sind mit unserer Formel nicht zu beschreiben.
Die gemessene Geschwindigkeit fällt stets geringer aus, als es unsere Zauberformel vorhersagt.  Problematisch ist die wachsende \textbf{relativistische Bedeutung} bei hohen Geschwindigkeiten.
\\ Sei hier ein kleiner Ausflug in die relativistische Welt erlaubt. Der relative Energiesatz beschreibt den Gesamtzustand eines System mit der Energie E, Impuls p, Vakuumlichtgeschwindigkeit c und der Ruheenergie $E_0$.
\begin{align} \label{eqn:relEnergiesatz}
E^2 = (pc)^2 + (E_0)^2
\end{align}

\subsubsection{Spezifische Ladung}
Zurück zur Beschreibung der Elektronen in der Fadenstrahlröhre. Bildet sich z.B. eine \textbf{Kreisbahn} aus (Für Schraubenbahnen gelten alle Formulierungen uneingeschränkt für den jeweiligen senkrechten Anteil der Geschwindigkeit).
Da die Elektronen wie durch \textit{Zauberhand} auf einer Bahn gehalten wird, nehmen wir an, dass die wirkende \textbf{Lorentzkraft und Zentripetalkraft} gerade ein Gleichgewicht bilden. Hinweis: $v \perp B$

\begin{align} \label{eqn:Ansatz Fadenstrahlröhre}
m \frac{v^2}{r} 	&= e \cdot v \cdot B \\ 
\frac{e}{m}			&= \frac{v}{r \cdot B} \qquad  \bigg|^2
\end{align}Durch den kleinen Trick des Quadrierens  kann $v^2$ durch \ref{eqn:ElektronenFadenstrahlGeschw} eliminiert werden. Dies führt uns zu 
\begin{align*}
\frac{e^2}{m^2} = \frac{2Ue}{m(rB)^2}
\end{align*}Nach Kürzen von e und m ergibt sich die Gleichung zur Bestimmung der \textbf{spezifischen Ladung} mit einer Fadenstrahlröhre.
\begin{align} \label{eqn:spezifischeLadung}
\boxed{\frac{e}{m} = \frac{2 \cdot U }{\left(r \cdot B \right)^2}}
\end{align}
\subsubsection{Larmorradius}
Wie schon bei der spezifischen Ladung bzw. Ausströmgeschwindigkeit der Elektronen lassen sich recht einfach gute Aussagen tätigen.  Der Radius ist kaum komplexer. 
Stellen wir \ref{eqn:Ansatz Fadenstrahlröhre} nach r um erhalten wir den Radius um denen sich die Elektronen gerade im Raum bewegen.
\begin{align} \label{eqn:Larmorradius}
\boxed{R_l = \frac{m \cdot v}{e \cdot B}}
\end{align} Und nennen diese charakteristische Größe \textbf{Larmorradius}. 

\subsubsection{Zyklotronfrequenz}
Bewegt sich etwas zyklisch, so ist die Frequenz interessant. Dass in unserem Beispiel die Elektronen eine hohe Umlauffrequenz haben dürften, entnehmen wir unseren Beobachtungen. Ein Flackern / Aufleuchten konnten wir nämlich nicht feststellen. 
Ist $v \perp r$ so gilt $\omega = \frac{v}{r}$. Für unsere Zentripetalkraft muss also gelten
\begin{align*}
m \, \omega^2 R_l = e \, v \, B
\end{align*}
\begin{align} \label{eqn:Zyklotronfrequenz}
\boxed{\omega_{Zykl} = \frac{e}{m} B}
\end{align}Wir stellen erstaunt fest, dass die Umlauffrequenz einzig von der spezifischen Ladung und dem B-Feld abhängig ist, jedoch nicht vom gewählten Radius!
morgen: Zyklotron als Beschleuniger 

\subsection{Halleffekt}
\begin{figure}\begin{center}
\includegraphics[scale=0.8]{img/Hall.png}\end{center}
\caption{Der Halleffekt wird durch die Verschiebung freier Leitungselektronen mit der Lorentzkraft hervorgerufen. Quelle:\textit{www.schule-bw.de} }
\label{pic:Halleffekt}
\end{figure}
Ist einer der elementaren Versuchen zur Untersuchung, dass Leiter tatsächlich  freie Ladungsträger besitzen. \par 
Nehmen wir zunächst eine dünne Metallplatte durchströmt vom konstanten Strom I. Gleichzeitig legen wir ein B-Feld an, das senkrecht Plattenebene steht. Wenn wir uns bildlich vorstellen, dass die freien Ladungsträger im B-Feld durch die Lorentzkraft (\ref{eqn:Lorentzkraft}) abgelenkt werden. Nach einiger Zeit sammeln sich vermehrt Elektronen an einer Seitenkante an, ein Potential bildet sich also aus. Dieses Potential nennen wir \textbf{Hallspannung}
\begin{align} \label{eqn:Hallspannung}
\boxed{U_H = R_H \cdot I}
\end{align}oder aber $U_H = A_H \cdot B \cdot \frac{I}{d}$. Wobei $R_H$ der Hallwiderstand, $A_H$ der Hallkoeffizient und d die Plattendicke ist. 
Die Hallspannung ist materialabhängig und  gemäß \ref{eqn:Hallspannung} umso größer, je höher der Hallwiderstand wird. Kupfer besitzt einen vergleichsweise kleinen Hallwiderstand von $-6 \cdot 10^{-5} \mathrm{\frac{cm^3}{As}}$, hingegen Bismut mit $R_H = -0.9\mathrm{\frac{cm^3}{As}}$ relativ hohe Hallspannungen erlaubt. Man setzt es daher gerne für die Magnetfeldbestimmung in \textbf{Hallsonden} ein.

\subparagraph{Herleitung}
Der Strom $I$ (j = const) ist definiert als
\begin{align}
I 			= \int \vec{j} \cdot d\vec{A} = j \int dA = j \cdot A = j \cdot b\,d\\
\boxed{j 	= \frac{I}{b\, d} = n (-e) v_{drift}} \label{eqn:stromdichte_Drift}
\end{align}

Lenkt das B-Feld die freien Ladungsträger mit der Geschwindigkeit $v_{drift}$ ab, so wirkt hier die Lorentzkraft. Nehme weiter an, dass $\vec{v}_{drift}$ und $\vec{B}$ orthogonal zueinander stehen.
\begin{align} \label{eqn:HallHerleitung LKraft}
\vec{F_L} &= q \left(\vec{v}_{drift}  \times \vec{B} \right) \qquad  \bigg| \vec{v}_{drift} \perp \vec{B} \\
F 			&= q \, v_{drift} \, B
\end{align}
Bis jetzt  nehmen wir an, dass die Lorentzkraft die einzig wirkende Kraft ist, durch die die Elektronen abgelenkt werden. Unsere Versuche konnten jedoch zeigen, dass die Hallsonde nicht komplett durchlässig für Elektronen wird und eine Spannung messbar ist. \par 
Nach einiger Zeit sammeln sich so viele Elektronen an einer jeweiligen Plattenkante an, dass es zur Ausbildung eines \textbf{E-Feldes} kommt. Dieses E-Feld wiederum lenkt Elektronen, die durch die Lorentzkraft abgelenkt werden wieder zurück. Man kann sich einen Hall-Leiter als kleinen Plattenkondensator mit angelegter Plattenspannung sowie schnellen Elektronen, die in einem B-Feld abgelenkt werden.
Es stellt sich nach einer gewissen Zeit ein Gleichgewicht zwischen  elektrischer Kraft und Lorentzkraft ein.
Mit $F_{el} = e\, E = \frac{U_H e}{d}$ und $F_L = e \, v_{drift} \, B$
\begin{align*}
F_{el} &= F_{L} \\
\frac{U_H e}{d} &= e \, v_{drift} \, B\\
U_H &= v_{drift} \, d \, B
\end{align*} Ersetzen wir nun $v_{drift}$ durch den Ausdruck in \ref{eqn:Stromdichte_Drift}. Dann bekommen wir

\begin{align}
\boxed{U_H = \frac{j}{n\, e} d \, B = \frac{I}{n \, e \,b} B }
\end{align}
Wenn wir uns an die Ohmsche Definition des Widerstands (\ref{Ohmsches Gesetz}) erinnern. Für Hallspannungen wird sich ergeben
\begin{align} \label{eqn:Hall Ohm}
\boxed{U_H = R_H I}
\end{align} Und tatsächlich können wir unsere Hallspannung in geschickte Faktoren unterteilen
\begin{align} \label{eqn:HallspannungDef}
\boxed{U_H = \underbrace{\frac{B}{n \, e \,b}}_{R_H} I}
\end{align}Wobei $\frac{1}{n\, e}$ eine Materialspezifische Größe bildet , $B, \,d$ hingegen beeinflussbar sind.

Anwendung findet der Halleffekt in der Messung von Magnetfeldern und der Elektronendichte $n$ von Stoffen.

\subsection{Magnetisches Moment}
\begin{align} \label{eqn:Magnetisches Moment1}
\vec{D} = I \vec{A} \times \vec{B}
\end{align} $\vec{A}$ ist hierbei der Flächennormalenvektor. Häufig ordnet man der vektoriellen Größe magnetisches Moment $I \,\vec{A}$  $\vec{\mu}$ bzw. $\vec{p_m}$ zu. So ergibt sich (wie beim elektrischen Moment)
\begin{align} \label{eqn:Magnetisches Moment}
\boxed{\vec{D} = \vec{p_m} \times \vec{B}}
\end{align}

\subsection{Biot-Savart Gesetz}
Um Elektrische Felder zu bestimmen, benutzen wir vorher die Integration über geeignete Teilflächen, danach lernten wir den Gaußschen Satz kennen, der einiges vereinfachte und uns die erste Maxwellsche Gleichung brachte.
Gleiches ist auch für B-Felder möglich. Hier wollen wir zunächst das B-Feld eines sehr kleinen Leiterstückchens betrachten, das an einem Ort $\vec{r}$ ein Magnetfeld $d\vec{B}$ hervorruft. 

Das Gesetz von Biot-Savart soll hier nicht hergeleitet werden, die Anwendung ist viel wichtiger.
\begin{align} \label{eqn:Biot-Savart}
\boxed{d\vec{B} = \frac{\mu_0}{4\pi} I \frac{d\vec{l} \times \vec{r}}{r^3}}
\end{align}$d\vec{l}$ gibt den Betrag des infinitesimalen Leiterstücks in tangentialer Stromrichtung an. $\frac{\vec{r}}{r^3} = \frac{\hat{e}_r}{r^2}$ Gibt die Abschwächung des Magnetfelds im Abstandsquadrat an.   Haben wir schon einmal $d\vec{B}$ bestimmt, ist es ein leichtes $\vec{B}$ zu 
ermitteln.
Durch Integration ergibt sich $\vec{B} = \int d\vec{b}$ 
Diese Beziehung erlaubt es uns jedes beliebige Magnetfeld einer beliebigen Stromverteilung zu beschreiben. Symmetrische Probleme lassen sich natürlich einfacher lösen, sofern man die Symmetrie und geeignete Koordinatensysteme nutzt.
\subparagraph{Beispiel: Magnetfeld eines stromdurchflossenen Leiters}
Noch einfügen

\subsection{Amperesches Gesetz} \label{sec:Amperesches Gesetz}
Häufig auch als \textit{Durchflutungsgesetz} bezeichnet, wir wollen hier jedoch zu Ehren André-Marie Ampères den Begriff des Ampereschen Gesetz gebrauchen. \\
Erinnern wir uns an die Gaußsche Formulierung für Elektrische Felder. Sie erleichterte uns die Berechnung $ \oint \vec{E} d\vec{A} = \tfrac{Q}{\epsilon_0}$.
Wir kennen mittlerweile, dass es keine magnetischen Monopole/Quellen geben kann. Eine Hüllfläche um eine magnetische Erzeugung zu legen ist also nicht zweckmäßig \footnote{Hieran scheiterten auch die alten Physiker, die nach den magnetischen Elementarteilchen suchten, in Analogie zur Elektrostatik} \par
Aus dem vorherigen Abschnitt kennen wir die Tatsache, dass Leiterströme (egal ob makroskopisch oder sinnbildliche Leiterströme durch einen Leiter erzeugt) um die Dipolachse ein Magnetfeld erzeugen. Dieses ist so gerichtet, dass es sich senkrecht zur Stromflussrichtung ausbildet und rotationssymmetrisch ist. Man verallgemeinert, es bilden sich \textbf{konzentrische Äquipotentiallinien} (hier nicht ganz richtig benutzt!)  um den Leiter mit dem Abstand R aus.
Ampère konnte zeigen, dass die Durchflutung (Strom des Leiters) \textbf{über den Rand einer Fläche} (aufgespannt von einer geschlossenen Feldlinie\footnote{Im Gegensatz zur Elektrostatik gibt es hier tatsächlich geschlossene Feldlinien}) integriert über B eine Proportionalität zum Strom I herstellt. Das Gesetz stellt eine Verbindung zwischen dem Strom durch den es hervorgerufen wurde und der magnetischen Feldstärke, die dier Strom erzeugt.
\begin{align} \label{eqn:Amperesches Gesetz} 
\oint \limits_{\partial S} \vec{B} \cdot d\vec{s} = \mu_0 \int \limits \vec{j\left( r \right)} \cdot d\vec{A}
\end{align}
Da $\vec{B} \, , d\vec{s}$ parallel zueinander orientiert sind, ist der eingeschlossene Winkel zwischen $\vec{B}$ und $d\vec{s}$ 0. Somit ist folglich das Skalarprodukt betragsmäßig gegeben durch $B \, ds$. Wir können unseren Term für eine leichtere Integration vereinfachen. Zusätzlich ist für $\vec{j\left( r \right)} = \, const.$ und so $I = j \cdot A$. Die Fläche A ist die Fläche, die durch den Rand der geschlossenen Feldlinie aufgespannt ist. So ist es insbesondere auch möglich \textbf{Magnetfelder in Leitern} zu bestimmen. Hierzu betrachtet man den eingeschlossenen Teil des Stroms.
\begin{align*}
\boxed{\oint \limits_{\partial S} B \cdot ds = \mu_0 I}
\end{align*}

\subparagraph{Anwendung} Das Gesetz von Biot-Savart ist zur allgemeinen Bestimmung von B-Feldern geeignet, jedoch nicht um einfache Felder durch symmetrische Objekte wie Kreise etc. zu berechnen. Hier kommt die Stärke des Ampereschen Gesetz. Es gilt für uns nur eine geeignete Symmetrie zu finden und so das Linienintegral zu lösen. Den Rest erledigt für uns die Macht des Integrals. 
\subparagraph{Kreissymmetrische Stromverteilung} Eine Kreissymmetrische Stromverteilung ist z.B. das B-Feld um einen unendlich lang ausgedehnten Leiter. Zugleich nehmen wir an, dass der Leiter unendlich dünn ist.
Untersuchen wir eine spezielle Feldlinie, die gerade ihren Kreismittelpunkt im Urpsrung des Koordinatensystems besitzt. Zudem sollte der Leiter in z-Richtung blicken.
Diese Überlegungen führen dazu, dass nur noch ein eindimensionales Kreisproblem vorliegt (das wäre schon anders, wenn wir den Leiter leicht kippen würde).
Um nun das B-Feld zu berechnen, das wir an der Äquipotentiallinie des Radius R haben, wollen wir zum ersten mal das Amperesche Gesetz (\ref{eqn:Amperesches Gesetz}) benutzen.
\begin{align*}
\oint \limits_{\partial S} \vec{B} \cdot d\vec{s} &= \mu_0 \int \limits \vec{j\left( r \right)} \cdot d\vec{A} \\
B \int \limits_{0}^{2\pi R}  ds &= \mu_0 I
\end{align*} Für unser B-Feld um einen \textbf{langen Leiter} ergibt sich also
\begin{align*}
B = \frac{\mu_0}{2\pi R} I
\end{align*}

\subparagraph{Spulen und Amperesches Gesetz} Für Spulen merkt man sich, dass jede einzelne Durchflutung einer Windung aufsummiert wird. Das Gesetz erweitert sich in diesem Fall für N Windungen zu
\begin{align} \label{eqn:Amperesches Gesetz für Spulen}
\oint \limits_{\partial S} B \cdot ds = \mu_0 \, N \, I 
\end{align}


\subparagraph{Was ist $\partial S$?} Um diese Frage zu klären wollen wir uns nochmal den \textbf{Stokes'schen Satz} anschauen. Er besagt für ein Vektorfeld $\vec{A}$ gilt
\begin{align} \label{eqn:Stokes}
\oint \limits_{\partial S} \vec{A} \cdot d\vec{l} = \int \limits_S \left(\nabla \times \vec{A} \right) \cdot d\vec{S}
\end{align} Es handelt sich bei $d\vec{S}$ um eine sehr kleines Flächenstück in Richtung des Flächennormalenvektors. $d\vec{l}$ hingegen ist ein Streckenstück, wobei $d\vec{l}$ stets tangential zum und auf dem Rand der Fläche verläuft. $\partial S$ ist bildlich gesprochen z.B. der Umfang eines Kreises. Dies mathematisch korrekt zu beschreiben ist sehr aufwendig. Uns soll hier eine einfache Konvention reichen. Immer wenn wir über den kompletten Rand einer Fläche integrieren wollen, zeigen wir dies durch folgende Konvention:$\oint \limits_{\partial S} \cdots d\vec{l}$.

\subsection{Magnetischer Fluss}
Im Abschnitt \ref{sec:Amperesches Gesetz} überlegten wir uns, dass es nicht sinnvoll ist eine Hüllkurve um magnetische Erzeuger zu legen. Wirft sich die Frage auf, ob es überhaupt einen magnetischen Fluss geben kann. Die Antwort ist von solch großer Bedeutung, dass wir ihr nochmal einen eigenen Abschnitt widmen wollen. 
Der Magnetische Fluss verknüpft die Größen der B und E-Felder.

Legen wir um eine veränderliche Leiterschleife mit einem erzeugendem B-Feld eine plane Fläche A. so ergibt sich der magnetische Fluss als
\begin{align} \label{eqn:magn. fluss}
\boxed{\Phi_{magn.} = \oint \limits \vec{B} \cdot d\vec{A}}
\end{align}
Gemessen wird der magnetische Fluss in $1\, T  \, m^2$ bzw. $1 \, Wb \; \mathrm{(Weber)}$

Für Spulen gilt schon wie im Ampereschen Gesetz (\ref{eqn:Amperesches Gesetz für Spulen}) die Poportionalität zur Windungsanzahl.
\subparagraph{Ausblick} Ändern sich solche B-Felder zeitlich, dann muss sich natürlich auch der magnetische Fluss ändern. Das alle kennen wir, wenn wir z.B. den belasteten  Netzstecker aus einer stromführenden Steckdose ziehen und eine Funkregen wahrnehmen. Dieses Phänomen vereint die ganze Vielfalt der Elektrodynamik (magn./el. Fluss, Faradaysches Induktionsgesetz)

\subsection{Materie im Magnetfeld}
\subsubsection{Induzierte Dipole} Diamagnetismus
\subsubsection{Permanente Dipole}
\subsubsection{Ferromagnetismus}
\subparagraph{Hysteresekurve}
\subparagraph{Steighöhe}

\section{Maxwellgleichungen}

\section{Elektromagnetische Felder}
\section{Wechselfelder}

\end{document}