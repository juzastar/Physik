%\documentclass[ngerman]{scrartcl}
\documentclass[%
	pdftex,%              PDFTex verwenden da wir ausschliesslich ein PDF erzeugen.
	a4paper,%             Wir verwenden A4 Papier.
	oneside,%             Einseitiger Druck.
	12pt,              %  Grosse Schrift, besser geeignet f�r A4.
	halfparskip,%         Halbe Zeile Abstand zwischen Abs�tzen.	
	%chapterprefix,%       Kapitel mit 'Kapitel' anschreiben.
	%headsepline,%         Linie nach Kopfzeile.
	footsepline,%         Linie vor Fusszeile.	
	%bibtotocnumbered,%    Literaturverzeichnis im Inhaltsverzeichnis nummeriert einf�gen.
	idxtotoc%             Index ins Inhaltsverzeichnis einfügen.
]{scrartcl}
% Titel, Autor und Datum
% zusäzliche Pakete einbinden
\usepackage[ngerman]{babel}

\usepackage[usenames,dvipsnames]{pstricks}
\usepackage{float}
\usepackage{epsfig}
\usepackage{pstricks}

%erweiterte Tabellen
\usepackage{tabularx}


\usepackage{fancybox}
\usepackage{amsmath}
\usepackage{amsfonts}
\usepackage{amssymb} 
\usepackage{nicefrac}

%erzeugt die Mengen $\mathds{R}^3$
\usepackage{dsfont}
%geschwungene Buchstaben \mathscr{F} 
\usepackage{mathrsfs} 

\usepackage[utf8x]{inputenc}
%\usepackage[T1]{fontenc}
\usepackage{ae}

\usepackage{setspace}
\usepackage[left=25mm,right=25mm,top=28mm,bottom=25mm,headheight=21pt,headsep=27pt,footskip=30pt]{geometry}
\usepackage{blindtext}


\usepackage{graphicx}

%Packet für anklickbare Links
\usepackage{hyperref}

\usepackage[scaled=.85]{luximono}

%\pagestyle{headings}

%
%  4. Stil der Überschriften auf normale Schrift.
%     Wir verwenden für die Überschriften den selben Font wie für den Text.
%
%\setkomafont{chapter}{\Huge}
\setkomafont{section}{\normalfont\bfseries\LARGE}
\setkomafont{title}{\normalfont\Large\bfseries\Large}
\setkomafont{subsection}{\normalfont\bfseries\large}
\setkomafont{subsubsection}{\normalfont\bfseries}
			



\setkomafont{sectioning}{\normalfont\bfseries}       % Titel mit Normalschrift
\setkomafont{captionlabel}{\normalfont\bfseries}     % Fette Beschriftungen 
\setkomafont{pagehead}{\normalfont\itshape}          % Kursive Seitentitel
\setkomafont{descriptionlabel}{\normalfont\bfseries} % Fette Beschreibungstitel


%#uhrzeit
\usepackage{scrtime}

%Kopf- und Fußzeile
\usepackage{fancyhdr}
\pagestyle{fancy}
%\fancyhf{}
 
%Kopfzeile mittig mit Kaptilname
%\fancyhead[C]{\nouppercase{\leftmark}}
%Linie oben
%\renewcommand{\headrulewidth}{0.5pt}
 
%Fußzeile links bzw. innen
\fancyfoot[L]{Erstellt durch Florian Diekmann}
%Fußzeile mittig (Seitennummer)
\fancyfoot[C]{\thepage}
%Linie unten
\renewcommand{\footrulewidth}{0.5pt}
 
% Fußzeile auf jeder Seite - auch Kapitel und Inhaltsverzeichnis
%\fancypagestyle{plain}{%
%   \fancyhf{}%
%   \fancyhead[C]{} %Kapitelname ausblenden
%   \fancyfoot[L]{Text muss hier stehen}
%   \renewcommand{\headrulewidth}{0.0pt} %obere Linie ausblenden
%   \fancyfoot[C]{\thepage}
%}
% anklickbare Links
\definecolor{darkblue}{rgb}{0,0,.5}
\hypersetup{pdftex=true}


%%% eigene Macros
%\newcommand{\1vec}[2]{\mathrel{\hat{\#1}_{\#2}} } %
\newcommand{\entspricht}{\ensuremath{\mathrel{\widehat{=}}}} 
\newcommand{\kreis}[1]{\unitlength1ex\begin{picture}(2.5,2.5)%
\put(0.75,0.75){\circle{2.5}}\put(0.75,0.75){\makebox(0,0){#1}}\end{picture}}
\newcommand{\grad}{\ensuremath{\mathrm{grad\,}}} 
\newcommand{\Fdiv}{\ensuremath{\mathrm{div\,}}} 