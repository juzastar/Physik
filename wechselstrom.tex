\newpage
\section{Wechselfelder}

\subsection{Stromerzeugung}
\paragraph{Wechselstromgenerator}
\begin{align} \label{eqn:Wechselstromgenerator}
U_{ind} = \omega B  A \cos \left( \omega \, t \right)
\end{align}

\paragraph{mittlere Leistungsabgabe}
\begin{align} \label{eqn:Leistung}
\bar{P} &= \frac{1}{T} \int \limits_0^T P_{el} \, dt \\
		&= \frac{1}{T} \int \limits_0^T U_0 I_0 \cos^2 \omega t dt \\
\boxed{\bar{P}	= \frac{1}{2} U_0 I_0}
\end{align}

\subsection{Wechselstromkreise}
Für eine rein induktive Last ist der Stromfluss gegenüber der Wechselspannung um $\frac{\pi}{2} $ verzögert.

\subsection{Wechselstrom als komplexe Größe}
Man wählt den Ansatz mit $U(t) = U_0 \exp (i \omega t) $.
Für die elementaren Bauelemente ergeben sich dann  die\textbf{ Impedanzen / Admittanzen} zu
\par \gesetz{\begin{align} \label{eqn:Wechselstromwiderstände}
Z_c &= \frac{1}{i \omega C} \\
Y_c &= i \omega C \\
Z_L &= i \omega L \\
Y_L &= \frac{1}{i \omega L}\\
\tan \phi &= \frac{Im(Z)}{Re(Z)}\\
|Z| &= \sqrt{Re(Z)^2 + Im(Z)^2}
\end{align}}

\subsection{Thomsonscher Schwingkreis}
Es ergibt sich für eine Reihenschaltung von R, C, L ein schwingkreis der sich ähnlich dem eines harmonischen Oszillators verhält.
Es ergibt sich die Differentialgleichung
\begin{align} \label{eqn:ThomsonDGL}
\ddot{I} + \frac{R}{L} + \frac{1}{LC}I = 0
\end{align}
Für solche eine Schwingungsgleichung ergibt sich eine Lösung wie im Falle des frei gedämpften harmonischen Oszillator mit $\gamma = \frac{R}{2L}$ und $\omega_0 = \sqrt{\frac{1}{LC}} $, sowie $\beta = \sqrt{\gamma^2 - \omega_0^2} = \sqrt{\frac{R^2}{4L^2} - \frac{1}{LC}}$
\subparagraph{Lösung der DGL nach I(t)}
\begin{align} \label{eqn:ThomsonDGLLoes}
I(t) = \exp \left(- \gamma t \right) \left[  A_1 \exp ( \beta t) + A_2 \exp (- \beta t)        \right]
\end{align}
es ergeben sich die 3 Fälle zwischen Dämpfung und Eigenfrequenz 
\begin{itemize}
\item schwache Dämpfung $\gamma < \omega_0 \rightarrow R^2 < \frac{4L}{C}$
\item starke Dämpfung $\gamma > \omega_0$
\item aperiodischer Grenzfall $\gamma = \omega_0$
\end{itemize}

\paragraph{Erzwungene Schwingung}
Ein Thomsonscher Schwingkreis mit U, R, C, L. Nach der Maschenregel führt uns dies zu
\begin{align} \label{eqn:gezThomsonDGL}
\boxed{\ddot{I} + \frac{R}{L} \dot{I} + \frac{1}{LC} I = \frac{U_0 \omega}{L} \sin \omega t}
\end{align}
Die wichtige Resonanzfrequenz in der Resonanzkatastrophe \footnote{bachte: Hier nehmen wir an, dass ein Leiter keinen Widerstand und damit auch keine Dämpfung besitzt} befindet sich bei $\gamma = 0 \qquad \omega_0 = \omega$

\subsection{Transformation}
\paragraph{Stromkreis 1}
Zwei Spulen mit gleichen Induktivitäten haben wir schon bei der gegenseitigen Induktion zweier Spulen gehabt. hier war die transformierte Spannung näherungsweise der Ausgangsspannung äquivalent
\begin{align} \label{eqn:Trafo_Strom1}
U_{ind} = - L_1 \ddt I = - N_1 \ddt \Phi_1 
\end{align}
\paragraph{Stromkreis 2}
Nehmen wir eine idealer Ankopplung der zweiten Spule durch ein Eisenjoch an, so ergibt sich, dass der magnetische Fluss beider Spulen gleich sein muss. Es gilt also $\Phi_1 = \Phi_2$. 
\begin{align} \label{eqn:Trafo_Strom2}
\ddt \Phi_1 &= \ddt \Phi_2\\
U_2 = N_2 \ddt \Phi_2 &= \frac{N_2}{N_1} U_{ind}
\end{align}
Die allgemeine Zauberformel findet man als Verhältnis beider Wicklungen und den Spannungen
\begin{align} \label{eqn:Trafo_Formel}
\boxed{\frac{U_1}{U_2} = - \frac{N_1}{N_2}}
\end{align}
